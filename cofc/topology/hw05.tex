\documentclass[11pt]{article}
\usepackage{common}
\usepackage{upgreek}

\begin{document}
\header{5}

% --------------------------------------------------------------
%                         Exercises
% --------------------------------------------------------------

\hwpart{1}

\begin{ex}{8}
  Let $A, B, A_\alpha$ denote subsets of a space $X$. Determine whether the
  following equations hold; if an equality fails, determine whether one of the
  inclusions $\supset$ or $\subset$ holds.
\end{ex}

\begin{p}{a}
  $\cl{A \cap B} = \cl A \cap \cl B$.
\end{p}

\begin{proof}
  Equality does not hold, however $\subset$ does hold. Suppose $x \in \cl{A \cap
  B}.$ Then for all neighborhoods $U$ of $x$, $U \cap (A \cap B) \neq
  \varnothing$, which implies that both $U \cap A \neq \varnothing$ and $U \cap
  B \neq \varnothing$. Therefore $x \in \cl A$ and $x \in \cl B$. Hence $\cl{A
  \cap B} \subset \cl A \cap \cl B$.
\end{proof}

\begin{p}{c}
  $\cl{A - B} = \cl A - \cl B$.
\end{p}
\begin{proof}
  Equality does not hold, however $\supset$ does hold. Suppose $x \in \cl A -
  \cl B$, so every neighborhood of $x$ intersects $A$ but there exists a
  neighborhood $V$ of $x$ disjoint from $B$. Suppose for contradiction that
  there exists a neighborhood $U$ of $x$ such that $U \cap (A - B) =
  \varnothing$. Then since $U \cap V \subset U$, we have $(U \cap V) \cap (A -
  B) = \varnothing$. Notice $U \cap V$ is also a neighborhood of $x$ and since $x \in \cl A$, we know $(U
  \cap V) \cap A \neq \varnothing$, so it must be the case that $(U \cap V) \cap
  B \neq \varnothing$. However, this implies $V \cap B \neq \varnothing$, which
  contradicts the assumption that $V$ is disjoint from $B$. Therefore all
  neighborhoods $U$ of $x$ intersect $A - B$, so that $x \in \cl{A - B}$.
\end{proof}

\begin{ex}{13}
  Show that $X$ is Hausdorff if and only if the \emph{diagonal} $\Delta = \{x
  \times x : x \in X\}$ is closed in $X \times X$.
\end{ex}
\begin{proof}
  $(\Rightarrow)$ Suppose $X$ is Hausdorff and let $x\times y$ be in the complement of
  the diagonal: $x\times y \in X \times X - \Delta$. Then $x \neq y$, so there
  exist disjoint neighborhoods $U$ of $x$ and $V$ of $y$. Note that because $U$
  and $V$ are disjoint, they share no equal elements, hence $U \times V$ does
  not intersect $\Delta$. Therefore $x \times y \in U \times V \subset X \times
  X - \Delta$. Since we have shown that arbitrary elements of the complement of
  the diagonal are elements of open sets contained within the complement of the
  diagonal, the complement of the diagonal is open. Hence the diagonal is
  closed.

  $(\Leftarrow)$ Suppose $\Delta$ is closed in $X \times X$. Let $x,y$ be
  distinct elements of $X$. Then $x \times y \notin \Delta$, so $x \times y \in
  X \times X - \Delta$ which is open. Thus there exists some basis element $U
  \times V$ such that $x \times y \in U \times V \subset X \times X - \Delta$,
  where $U, V$ are open in $X$. However, since $U \times V$ is contained in the
  complement of the diagonal, we see that $U$ and $V$ must be disjoint. We have
  shown that we can find disjoint neighborhoods around any two distinct elements
  in $X$, therefore $X$ is Hausdorff.
\end{proof}

\begin{ex}{16.b}
  Consider the five topologies on $\mathbb{R}$ given in Exercise 13.7. Which of
  these topologies satisfy the Hausdorff axiom? the $T_1$ axiom?
\end{ex}
\begin{solution}
  Recall these are
  \begin{align*}
    \mathcal{T}_1 &= \text{ the standard topology } \\
    \mathcal{T}_2 &= \text{ the topology of } \mathbb{R}_K \\
    \mathcal{T}_3 &= \text{ the finite complement topology } \\
    \mathcal{T}_4 &= \text{ the upper limit topology, having all sets }
                (a, b]
                \text { as basis} \\
    \mathcal{T}_5 &= \text{ the topology having all sets }
                (-\infty, a) = \{x : x < a\}
                \text { as basis}.
  \end{align*}

  $\mathcal{T}_1, \mathcal{T}_2, \mathcal{T}_3, \mathcal{T}_4$ satisfy the $T_1$
  axiom and $\mathcal{T}_1, \mathcal{T}_2, \mathcal{T}_4$ satisfy the Hausdorff
  axiom.
\end{solution}

\begin{ex}{18}
  Determine the closures of the following subsets of the ordered square:
  \begin{align*}
    A &= \{(\frac{1}{n}) \times 0 : n \in \mathbb{Z}_+\} \\
    B &= \{(1 - \frac{1}{n}) \times \frac{1}{2} : n \in \mathbb{Z}_+\} \\
    C &= \{x \times 0 : 0 < x < 1\} \\
    D &= \{x \times \frac{1}{2} : 0 < x < 1\} \\
    E &= \{\frac{1}{2} \times y : 0 < y < 1\}
  \end{align*}
\end{ex}

\begin{solution}
  \begin{align*}
    \cl A &= A \cup \{0\times1\} \\
    \cl B &= B \cup \{1 \times 0\} \\
    \cl C &= \{x \times 1 : 0 \le x < 1\} \cup \{x \times 0 : 0 < x \le 1\}\\
    \cl D &= D \cup \cl C \\
    \cl E &= \Big[\frac{1}{2} \times 0, \frac{1}{2} \times 1\Big]
  \end{align*}
\end{solution}

\hwpart{2}

\noindent Let $X$ be a topological space. Let $A \subset X$ be a subset.

\begin{p}{a}
  Show that $\bd (\bd A) \subset \bd A$.
\end{p}
\begin{proof}
  Notice that
  \[ \bd (\bd A) \;=\; \cl{\bd A} \cap \cl{X - \bd A} \;\subset\; \cl{\bd A}
  \;=\; \bd A.\]
\end{proof}

\begin{p}{b}
  Find an example in which $\bd (\bd A) \neq \bd A$.
\end{p}
\begin{solution}
  When $A = \mathbb{Q} \subset \mathbb{R}$, we have $\bd \mathbb{Q} = \mathbb{R}$ and
  $\bd (\bd \mathbb{Q}) = \bd \mathbb{R} = \varnothing$.
\end{solution}

\begin{p}{c}
  Show that $\bd (\bd (\bd A)) = \bd (\bd A)$.
\end{p}
\begin{proof}
  First note that considering $\bd A$ as a subset of $X$, part (a) immediately
  tells us that $\bd (\bd (\bd A)) \subset \bd (\bd A)$. For the reverse
  direction, suppose $x \in \bd (\bd A)$. To show that $x \in \bd (\bd (\bd
  A))$, we will simply rule out the alternative possibilities that
  $x \in \intr (\bd (\bd A))$ or $x \in \extr (\bd (\bd A))$.

  So suppose for contradiction that $x \in \intr (\bd (\bd A))$. Then there
  exists a neighborhood $U$ of $x$ such that $U \subset \bd (\bd A)$. Since
  \[ \bd (\bd A) \;=\; \cl{\bd A} \cap \cl{X - \bd A} \;=\; \bd A \cap \cl{X - \bd
  A} \;\subset\; \bd A,\]
  we see that $U \subset \bd A$. Thus $x \in \intr (\bd A)$, which contradicts
  our assumption that $x \in \bd (\bd A)$.

  Next suppose for contradiction that $x \in \extr (\bd (\bd A))$. Then there
  exists a neighborhood $U$ of $x$ such that $U \subset X - \bd (\bd A)$. But
  then $x \in X - \bd (\bd A)$ which contradicts our assumption that $x \in \bd
  (\bd A)$.

  Therefore the only possibility is that $x \in \bd (\bd (\bd A))$, and we
  conclude that \\ $\bd (\bd (\bd A)) = \bd (\bd A)$.
\end{proof}

\end{document}
