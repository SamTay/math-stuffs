\documentclass[11pt]{article}
\usepackage{common}

\begin{document}
\header{2}

% --------------------------------------------------------------
%                         Exercises
% --------------------------------------------------------------

\hwpart{1}

\begin{ex}{2.1}
  Let $f:A \to B$. Let $A_0 \subset A$ and $B_0 \subset B$.
\end{ex}

\begin{p}{a}
  Show that $A_0 \subset f^{-1}(f(A_0))$ and that equality holds if $f$ is injective.
\end{p}

\begin{proof}
$(\subset)$ Notice that
  \begin{align}
    a \in A_0
      &\implies f(a) \in f(A_0) \label{im0} \\
      &\implies a \in f^{-1}(f(A_0)) \label{pre0}
  \end{align}
  where \eqref{im0} follows from the definition of image and \eqref{pre0}
  follows from the definition of preimage. Therefore $A \subset
  f^{-1}(f(A_0))$.

  $(\supset)$ Suppose further that $f$ is injective. Then
  \begin{align}
    a \in f^{-1}(f(A_0))
      &\implies f(a) \in f(A_0) \label{pre-1} \\
      &\implies f(x) = f(a) \text{ for some } x \in A_0 \label{im-1} \\
      &\implies a = x \in A_0 \label{inj-0}
  \end{align}
  where \eqref{pre-1} and \eqref{im-1} are simply from the definitions of
  preimage and image respectively, while \eqref{inj-0} follows from the
  injectivity of $f$. Therefore $f^{-1}(f(A_0)) \subset A_0$. Of
  course, since we've already shown that $A_0 \subset f^{-1}(f(A_0))$ in
  general, we conclude that injectivity of $f$ implies $A_0 = f^{-1}(f(A_0))$.

  % ORIGINAL PROOF
  %$(\subset)$ Let $x \in A_0$. By definition of image, $f(x) \in f(A_0)$. By
  %definition of preimage, $f^{-1}(f(A_0)) = \{ a \mid f(a) \in f(A_0)$, and it
  %follows immediately that $x \in f^{-1}(f(A_0))$. Therefore $A \subset f^{-1}(f(A_0))$.

  %$(\supset)$ Suppose further that $f$ is injective and let $x \in
  %f^{-1}(f(A_0))$. By definition of preimage, this means that $f(x) \in
  %f(A_0)$, and by definition of image it follows that there exists $a \in A_0$
  %such that $f(x) = f(a)$.  Since $f$ is injective, it must be the case that $x
  %= a$, and thus $x \in A_0$. Therefore $f^{-1}(f(A_0)) \subset A_0$. Of
  %course, since we've already shown that $A_0 \subset f^{-1}(f(A_0))$ in
  %general, we conclude that injectivity of $f$ implies $A_0 = f^{-1}(f(A_0))$.
  % ORIGINAL PROOF

\end{proof}

\begin{p}{b}
  Show that $f(f^{-1}(B_0)) \subset B_0$ and that equality holds if $f$ is surjective.
\end{p}

\begin{proof}
  $(\subset)$ Let $b \in f(f^{-1}(B_0))$. By definition of image,
  $$f(f^{-1}(B_0)) = \{ y \mid y = f(x) \text{ for at least one } x \in f^{-1}(B_0)\}$$
  so there must exist some $a \in f^{-1}(B_0)$ such that $b = f(a)$.
  By definition of preimage,
  $$ f^{-1}(B_0) = \{ x \mid f(x) \in B_0 \}.$$
  Thus, since $ a \in f^{-1}(B_0)$, we have $f(a) \in B_0$, and recalling that
  $b = f(a)$, we have $b \in B_0$. Therefore $f(f^{-1}(B_0)) \subset B_0$.

  $(\supset)$ Suppose further that $f$ is surjective and let $b \in B_0$. Since
  $f$ is surjective, there exists $a \in A$ such that $f(a) = b$. By definition
  of preimage, $f^{-1}(B_0) = \{ x \mid f(x) \in B_0 \}$, and since $f(a) \in
  B_0$, it is clear that $a \in f^{-1}(B_0)$. By definition of image,
  $$ f(f^{-1}(B_0)) = \{ y \mid y = f(x) \text{ for at least one } x \in f^{-1}(B_0) \}.$$
  Since $b = f(a)$ and $a \in f^{-1}(B_0)$, it is evident that $b \in
  f(f^{-1}(B_0))$. Therefore $B_0 \subset f(f^{-1}(B_0))$. Since we've already
  shown that $f(f^{-1}(B_0)) \subset B_0$ in general, we conclude that
  surjectivity of $f$ implies $f(f^{-1}(B_0)) = B_0$.

\end{proof}

\begin{ex}{2.2}
  Let $f: A \to B$ and let $A_i \subset A$ and $B_i \subset B$ for $i=0$ and $i=1$.
\end{ex}

\begin{p}{c}
  Show that $f^{-1}(B_0 \cap B_1) = f^{-1}(B_0) \cap f^{-1}(B_1)$.
\end{p}

\begin{proof}
  We can deduce rather mechanically that
  \begin{align}
    a \in f^{-1}(B_0 \cap B_1)
      &\iff f(a) \in B_0 \cap B_1 \label{pre1} \\
      &\iff f(a) \in B_0 \text{ and } f(a) \in B_1 \label{int1} \\
      &\iff a \in f^{-1}(B_0) \text{ and } a \in f^{-1}(B_1) \label{pre2} \\
      &\iff a \in f^{-1}(B_0) \cap f^{-1}(B_1) \label{int2}
  \end{align}
  where \eqref{pre1} and \eqref{pre2} follow from the definition of preimage and
  \eqref{int1} and \eqref{int2} follow from the definition of intersection. Therefore
  $f^{-1}(B_0 \cap B_1) = f^{-1}(B_0) \cap f^{-1}(B_1)$.
\end{proof}

\begin{p}{d}
  Show that $f^{-1}(B_0 - B_1) = f^{-1}(B_0) - f^{-1}(B_1)$.
\end{p}

\begin{proof}
  Observe that
  \begin{align}
    a \in f^{-1}(B_0 \cap B_1)
      &\iff f(a) \in B_0 - B_1 \label{pre3} \\
      &\iff f(a) \in B_0 \text{ and } f(a) \notin B_1 \label{min1} \\
      &\iff a \in f^{-1}(B_0) \text{ and } a \notin f^{-1}(B_1) \label{pre4} \\
      &\iff a \in f^{-1}(B_0) - f^{-1}(B_1) \label{min2}
  \end{align}
  where \eqref{pre3} and \eqref{pre4} follow from the definition of preimage and
  \eqref{min1} and \eqref{min2} follow from the definition of set difference. Therefore
  $f^{-1}(B_0 - B_1) = f^{-1}(B_0) - f^{-1}(B_1)$.
\end{proof}

\begin{p}{f}
  Show that $f(A_0 \cup A_1) = f(A_0) \cup f(A_1)$.
\end{p}

\begin{proof}
  Again, we construct equivalences:
  \begin{align}
    b \in f(A_0 \cup A_1)
      &\iff b = f(a) \text{ for some } a \in A_0 \cup A_1 \label{im1} \\
      &\iff b = f(a) \text{ for some } a \in A_0 \text{ or } a \in A_1 \label{un1} \\
      &\iff b \in f(A_0) \text{ or } b \in f(A_1) \label{im2} \\
      &\iff b \in f(A_0) \cup f(A_1) \label{un2}
  \end{align}
  where \eqref{im1} and \eqref{im2} follow from the definition of image and
  \eqref{un1} and \eqref{un2} follow from the definition of union. Therefore
  $f(A_0 \cup A_1) = f(A_0) \cup f(A_1)$.
\end{proof}

\begin{p}{g}
  Show that $f(A_0 \cap A_1) \subset f(A_0) \cap f(A_1)$ and that equality
  holds if $f$ is injective.
\end{p}

\begin{proof}
  $(\subset)$ First note that
  \begin{align}
    b \in f(A_0 \cap A_1)
      &\implies b = f(a) \text{ for some } a \in A_0 \cap A_1 \label{im3} \\
      &\implies b = f(a) \text{ for some } a \in A_0 \text{ and } a \in A_1 \label{in1} \\
      &\implies b \in f(A_0) \text{ and } b \in f(A_1) \label{im4} \\
      &\implies b \in f(A_0) \cap f(A_1) \label{in2}
  \end{align}
  and therefore $f(A_0 \cap A_1) \subset f(A_0) \cap f(A_1)$.

  $(\supset)$ Suppose further that $f$ is injective. Then
  \begin{align}
    b \in f(A_0) \cap f(A_1)
      &\implies b \in f(A_0) \text{ and } b \in f(A_1) \nonumber \\
      &\implies
        \exists\; a_0 \in A_0 \mid f(a_0) = b
        \quad \text{ and } \quad
        \exists a_1 \in A_1 \mid f(a_1) = b \nonumber \\
      &\implies b = f(a) \text{ for some } a \in A_0 \cap A_1 \label{inj3} \\
      &\implies b \in f(A_0 \cap A_1) \label{im6} \nonumber
  \end{align}
  where \eqref{inj3} follows from the injectivity of $f$, since we must have
  $a_0 = a_1$. Therefore when $f$ is injective we have $f(A_0 \cap A_1) =
  f(A_0) \cap f(A_1)$.
\end{proof}

\begin{p}{h}
  Show that $f(A_0 - A_1) \supset f(A_0) - f(A_1)$ and that equality holds when
  $f$ is injective.
\end{p}

\begin{proof}
  $(\supset)$ Let $b \in f(A_0) - f(A_1)$. Then there exists $a_0 \in A_0$ such that $f(a_0) = b$, but $f(a_1) \neq b$ for all $a_1 \in A_1$. Hence $a_0 \notin A_1$, so we have $f(a_0) = b$ for $a_0 \in A_0 - A_1$. Thus $b \in f(A_0 - A_1)$ and we conclude that $f(A_0 - A_1) \supset f(A_0) - f(A_1)$.

  $(\subset)$ Suppose further that $f$ is injective and let $b \in f(A_0 -
  A_1)$. Then there exists $a \in A_0 - A_1$ such that $f(a) = b$. Note that,
  since $a \in A_0$ and $a \notin A_1$, the injectivity of $f$ implies that
  there can be no $a_1 \in A_1$ such that $f(a_1) = b$. Therefore $b \in
  f(A_0)$ and $b \notin f(A_1)$, or equivalently, $b \in f(A_0) - f(A_1)$. Thus
  $f(A_0 - A_1) \subset f(A_0) - f(A_1)$, and we conclude that when $f$ is
  injective, $f(A_0 - A_1) = f(A_0) - f(A_1)$.
\end{proof}

\begin{ex}{3.12}
  Let $\mathbb{Z}_+$ denote the positive integers. Consider the following order relations on $\mathbb{Z}_+ \times \mathbb{Z}_+$:
  \begin{enumerate}[(i)]
    \item The dictionary order
    \item $(x_0 \times y_0) < (x_1 \times y_1)$ if either $x_0 - y_0 < x_1 -
      y_1$, or $x_0 - y_0 = x_1 - y_1$ and $y_0 < y_1$
    \item $(x_0 \times y_0) < (x_1 \times y_1)$ if either $x_0 + y_0 < x_1 +
      y_1$, or $x_0 + y_0 = x_1 + y_1$ and $y_0 < y_1$
  \end{enumerate}
  In these order relations, which elements have immediate predecessors? Does the set have a smallest element? Show that all three order types are different.
\end{ex}

\begin{solution}

  \phantombackup{what}

  \begin{enumerate}[(i)]
    \item All elements in $\{ (n \times m) \in \mathbb{Z}_+ \times \mathbb{Z}_+
      \mid m > 1 \}$, i.e. elements above the ``first row", have immediate
      predecessors. The smallest element is $(1 \times 1)$.
    \item All elements in $\{ (n \times m) \in \mathbb{Z}_+ \times \mathbb{Z}_+
      \mid n > 1, m > 1 \}$, i.e. elements other than the ``first row" and
      ``first column", have immediate predecessors. There is no smallest
      element.
    \item All elements other than $(1 \times 1)$ have immediate predecessors.
      The smallest element is $(1 \times 1)$.
  \end{enumerate}

  To show that all three order types are different, it suffices to point
  out their different order type characteristics. For example, we know that
  (ii) is different from (i) and (iii) because (ii) has no smallest element,
  while (i) and (iii) do. But (i) and (iii) are also different from each other
  because there are infinitely many elements without immediate predecessors in
  (i), and only one such element in (iii).

  Of course, to be pedantic, we could assume for contradiction that there
  exists an order preserving bijection $f$ between these order relations. It
  then becomes clear that $f$ preserves the differing characteristics just
  mentioned -- hence the contradiction.
\end{solution}

\begin{ex}{3.15 (b)}
  Assume $\mathbb{R}$ has the least upper bound property. Does $[0, 1] \times [0, 1]$ in the dictionary order have the least upper bound property? What about $[0, 1] \times [0, 1)$? What about $[0,1) \times [0, 1]$?
\end{ex}

\begin{solution}
  Both $[0, 1] \times [0, 1]$ and $[0,1) \times [0, 1]$ have the least upper bound
  property, however $[0, 1] \times [0, 1)$ does not. We will prove each of these in turn.
  First, let us define
  \mbox{$\pi_1, \pi_2 : \mathbb{R} \times \mathbb{R} \to \mathbb{R}$}
  by $\pi_1(x \times y) = x$ and $\pi_2(x \times y) = y.$

  Let $A = [0, 1] \times [0, 1]$ and $B \subset A$ be nonempty and bounded above by $z \in
  A$. Notice that $\pi_1(B)$ is a nonempty subset of $[0,1]$ that is bounded above by
  $\pi_1(z) \in [0, 1]$. By part (a), $[0,1]$ has the least upper bound
  property,\footnote{I am using the lemma proven in 3.15(a), as it declutters these
  arguments in 3.15(b).}
  so there exists $n
  = \text{lub } \pi_1(B) \in [0, 1]$. Next, consider the subset $C = \{
  (x \times y \in B \mid x = n\}$.
  \begin{description}
    \item[Case 1: $C = \varnothing$.]
      In this case, there is no element in $B$ with first coordinate equal to $n$. It
      follows that $(n \times 0)$ is an upper bound for $B$. To see that $(n \times 0)$ is
      the least such upper bound, suppose $(x \times y) \in A$ is another upper bound for $B$.
      Then $x \in [0, 1]$ is an upper bound for $\pi_1(B)$, and
      since $n = \text{lub}\, \pi_1(B)$, we have $n \leq x$. Of course this implies
      that $(n \times 0) \leq (x \times y)$. Therefore, in this case, $B$ has a least
      upper bound.
    \item[Case 2: $C \ne \varnothing$.]
      Similar to what we did for the first coordinate, notice that $\pi_2(C)$ is a nonempty
      subset of $[0, 1]$ that is bounded above by $1 \in [0, 1]$, and by part (a) we know
      that $m = \text{lub } \pi_2(C)$ exists. It is evident that $(n, m)$ is an upper
      bound for $B$. To see that $(n \times m)$ is the least upper bound, suppose $(x \times
      y)$ is another upper bound for $B$. Then, as reasoned in Case 1, $n \leq x$. If $n
      < x$ then we immediately have $(n \times m) < (x \times y)$. If $n = x$, then $(x
      \times y) = (n \times y)$, and since ($x \times y$) is an upper bound for $C$, which
      only has first coordinates equal to $n$, it is clear that
      $y$ must be an upper bound for $\pi_2(C)$. Of course, since $m = \text{lub
      }\pi_2(C)$, it follows that $m \le y$
      and hence $(n \times m) \le (x \times y)$. Therefore, in this case, $B$ has a least
      upper bound.
  \end{description}
  We have shown that in all cases, an arbitrary nonempty subset of $[0, 1] \times [0,
  1]$ that is bounded above has a least upper bound. We conclude that $[0, 1] \times [0,
  1]$ has the least upper bound property.

  Instead of repeating myself, I'll note that the proof for $[0,1) \times [0, 1]$ is
  identical; just notice that when using part (a), we leverage the fact that $[0,1)$ has
  the least upper bound property instead of $[0,1]$, so that $n = \text{lub } \pi_1(B) \in
  [0, 1)$ and $m = \text{lub } \pi_2(C) \in [0, 1)$.

  To see that $A = [0, 1] \times [0, 1)$ does not have the least upper bound property,
  consider the nonempty subset $B = \{ (0 \times y) \mid y \in [0, 1) \}$, which is
  bounded above by $(1 \times 0) \in A$. Let
  $(n \times m) \in A$ be an upper bound for $B$. Then it
  must be the case that $0 < n$. This is because the second coordinate in $[0, 1)$ is not
  bounded above.  That is, for any potential upper bound $(0 \times m) \in A$, we can pick
  some $y \in [0, 1)$ such that $m < y < 1$. Then $(0 \times m) < (0 \times y)$, so $(0
  \times m)$ is not an upper bound for $B$. Now since $0 < n \le 1$, there exists $\frac{n}{2}
  \in [0, 1]$ such that $0 < \frac{n}{2} < n$. Now $(\frac{n}{2} \times 0)$ is
  an upper bound for $B$ and yet $(\frac{n}{2} \times 0) < (n \times m)$. We have shown
  that, given any upper bound for $B$, there exists a lesser upper bound for $B$, and
  therefore $B$ has no least upper bound.
\end{solution}

\hwpart{2}

\begin{ex'}
  Let $B$ be a set that isn't the empty set. Define a function
  $$f : (\mathscr{P}(\mathscr{P}(B)) - \{\varnothing\}) \to \mathscr{P}(B)
  \;\text{ by }\;
  f(\mathcal{A}) = \bigcap_{A \in \mathcal{A}} A.$$
\end{ex'}

\begin{p}{a}
  Is $f$ injective? surjective?
\end{p}

\begin{solution}
  No, $f$ is not injective. Since $B$ is nonempty, there exist at least two
  subsets $\varnothing, \{b\} \in \mathscr{P}(B).$ These two subsets give rise
  to two distinct collections $\{\varnothing\}, \{\varnothing, \{b\}\}$ in the domain of
  $f$, however
  \[
    f(\{\varnothing\}) \;= \bigcap_{A \in \{\varnothing\}} A \;=\; \varnothing \label{eq1}
  \]
  and
  \[
  f(\{\varnothing, \{b\}\})
    \;= \bigcap_{A \in \{\varnothing, \{b\}\}} A
    \;=\; \varnothing \cap \{b\} \;=\; \varnothing \label{eq2}
  \]
  and therefore $f$ is not injective.

  Yes, $f$ is surjective. Let $C \in \mathscr{P}(B)$ be any subset of $B$. Then
  the collection $\{C\}$ exists in the domain $\mathscr{P}(\mathscr{P}(B)) -
  \{\varnothing\}$ such that
  $$ f(\{C\}) \;= \bigcap_{A \in \{C\}} A \;=\; C $$
  and therefore $f$ is surjective.
\end{solution}

\begin{p}{b}
  For $\mathcal{A}_1, \mathcal{A}_2 \in (\mathscr{P}(\mathscr{P}(B)) -
  \{\varnothing\})$, is it true that $f(\mathcal{A}_1) \cup f(\mathcal{A}_2) =
  f(\mathcal{A}_1 \cup \mathcal{A}_2)$? Notice this is different than Exercise
  2.2(f). There, $f$ is being applied to subsets of the domain, while here it's
  being applied to elements of the domain.
\end{p}

\begin{solution}
  No, this statement is not true in general. For any nonempty set $B$, there
  exists $b \in B$ such that $\varnothing, \{b\} \in \mathscr{P}(B)$ and
  $$\{\varnothing\}, \{\{b\}\} \;\in\; \mathscr{P}(\mathscr{P}(B)) -
  \{\varnothing\}.$$
  However, notice that
  \begin{align*}
    f\big(\{\varnothing\}\big) \cup f\big(\{\{b\}\}\big)
      &= \varnothing \cup \{b\} \\
      &= \{b\}
  \end{align*}
  and yet
  \begin{align*}
    f\big(\{\varnothing\} \cup \{\{b\}\}\big)
      &= f\big(\{\varnothing, \{b\}\}\big) \\
      &= \varnothing \cap \{b\} \\
      &= \varnothing.
  \end{align*}
  Therefore for any nonempty set $B$ there will exist $\mathcal{A}_1,
  \mathcal{A}_2 \in (\mathscr{P}(\mathscr{P}(B)) -
  \{\varnothing\})$ such that $f(\mathcal{A}_1) \cup f(\mathcal{A}_2) \neq
  f(\mathcal{A}_1 \cup \mathcal{A}_2)$.
\end{solution}

\end{document}
