\documentclass[11pt]{article}
\usepackage{common}
\usepackage{upgreek}

\begin{document}
\header{3}

% --------------------------------------------------------------
%                         Exercises
% --------------------------------------------------------------

\hwpart{1}

\begin{ex}{13.5}
  Show that if $\mathcal{A}$ is a basis for a topology on $X$, then the topology
  generated by $A$ equals the intersection of all topologies on $X$ that contain
  $\mathcal{A}$. Prove the same if $\mathcal{A}$ is a subbasis.
\end{ex}

\begin{proof}
  Let $\mathcal{T}$ be the topology generated by $\mathcal{A}$
  and $\{\mathcal{T}_\beta\}_{\beta \in H}$ be the collection of all topologies on $X$ that
  contain $\mathcal{A}$.

  $(\subset)$ Obviously $\mathcal{A} \subset \mathcal{T}$ so $\mathcal{T} =
  \mathcal{T}_\beta$ for some $\beta \in H$ and it must be the case that $\cap_{\beta \in
  H}\mathcal{T}_\beta \subset \mathcal{T}$.

  $(\supset)$ Conversely suppose $U \in \mathcal{T}$. By Lemma 13.1,
  $U$ is a union of basis elements in $\mathcal{A}$.
  But for any $\beta \in H$ we know $\mathcal{T}_\beta$ contains $\mathcal{A}$
  and is closed under unions. Hence $U \in \cap_{\beta \in
  H}\mathcal{T}_\beta$ and we conclude $\mathcal{T} \subset \cap_{\beta \in H}\mathcal{T}_\beta$.

  Therefore the topology $\mathcal{T}$ generated by $\mathcal{A}$ equals the
  intersection of all topologies on $X$ that contain $\mathcal{A}$.
\end{proof}

\begin{proof}
  Suppose $\mathcal{A}$ is a subbasis for a topology $\mathcal{T}$ on $X$. Let
  $\{\mathcal{T}_\beta\}_{\beta \in H}$ be the collection of topologies on
  $X$ that contain $\mathcal{A}$.

  $(\subset)$ Just as in the preceding proof, note that since
  $\mathcal{T}$ contains $\mathcal{A}$, $\mathcal{T} = \mathcal{T}_\beta$ for
  some $\beta \in H$, and therefore $\cap_{\beta \in
  H}\mathcal{T}_\beta \subset \mathcal{T}$.

  $(\supset)$ Conversely suppose $U \in \mathcal{T}$. Then $U$ is a union of
  finite intersections of elements of $\mathcal{A}$. But for any $\beta \in H$
  we know $\mathcal{T}_\beta$ contains $\mathcal{A}$ and is closed under finite
  intersections and arbitrary unions. Hence $U \in \cap_{\beta \in
  H}\mathcal{T}_\beta$ and we conclude $\mathcal{T} \subset \cap_{\beta \in
H}\mathcal{T}_\beta$.

  Therefore the topology $\mathcal{T}$ generated by the subbasis $\mathcal{A}$ equals the
  intersection of all topologies on $X$ that contain $\mathcal{A}$.
  \end{proof}

% TODO look this one over!
\begin{ex}{13.7}
  Consider the following topologies on $\mathbb{R}$:
  \begin{align*}
    \mathcal{T}_1 &= \text{ the standard topology } \\
    \mathcal{T}_2 &= \text{ the topology of } \mathbb{R}_K \\
    \mathcal{T}_3 &= \text{ the finite complement topology } \\
    \mathcal{T}_4 &= \text{ the upper limit topology, having all sets }
                (a, b]
                \text { as basis} \\
    \mathcal{T}_5 &= \text{ the topology having all sets }
                (-\infty, a) = \{x : x < a\}
                \text { as basis}.
  \end{align*}
  Determine, for each of these topologies, which of the others it contains.
\end{ex}

\begin{p}{1}
  $\mathcal{T}_1 \supset \mathcal{T}_3, \mathcal{T}_5$
\end{p}
\begin{p}{2}
  $\mathcal{T}_2 \supset \mathcal{T}_1, \mathcal{T}_3, \mathcal{T}_5$
\end{p}
\begin{p}{3}
  $\mathcal{T}_3$ does not contain any of the others
\end{p}
\begin{p}{4}
  $\mathcal{T}_4 \supset \mathcal{T}_1, \mathcal{T}_2, \mathcal{T}_3, \mathcal{T}_5$
\end{p}
\begin{p}{5}
  $\mathcal{T}_5$ does not contain any of the others
\end{p}

Most of these relations are very straightforward to verify. One that I feel
needs justification is $\mathcal{T}_2 \subset \mathcal{T}_4$. Consider an
arbitrary basis element of $\mathcal{T}_2$ which has the form $(a, b)$ or $(a,
b) - K$. If $x \in (a, b)$ then obviously $x \in (a, x] \subset (a, b)$. If $x
\in (a, b) - K$, we need to be more careful because we're not guaranteed that
$(a, x] \subset (a, b) - K$. However, in this case we can choose the least $n$
such that $\frac{1}{n} < x$ and pick a real number $c$ such that $\frac{1}{n} <
c < x$. Then $x \in (c, x] \subset (a, b) - K$. By Lemma 13.3 we conclude that
$\mathcal{T}_2 \subset \mathcal{T}_4$.

\begin{ex}{13.8 (a)}
  Apply Lemma 13.2 to show that the countable collection
  \[\mathcal{B} = \{(a,b) : a < b, a \text{ and } b \text{ rational}\} \]
  is a basis that generates the standard topology on $\mathbb{R}$.
\end{ex}

\begin{proof}
  Let $\mathcal{T}$ denote the standard topology on $\mathbb{R}$. First note
  that $\mathcal{B}$ is clearly a collection of open sets in $\mathcal{T}$.
  Next let $U \in \mathcal{T}$ and $x \in U$. Since
  \[ \mathcal{C} = \{ (a, b): a < b, a \text{ and } b \text{ real} \} \]
  is a basis for $\mathcal{T}$ we know that $U$ is a union of open intervals in
  $\mathbb{R}$. It follows that $x \in (a, b) \subset U$ for some real numbers
  $a, b$. Since $\mathbb{Q}$ is dense in $\mathcal{R}$ we can find rational
  numbers $p, q$ such that
  \[ a < p < x < q < b \]
  so $x \in (p, q) \subset (a, b) \subset U$. By Lemma 13.2, $\mathcal{B}$ is a basis for
  $\mathcal{T}$.
\end{proof}

\begin{ex}{13.8 (b)}
  Show that the collection
  \[\mathcal{C} = \{[a,b) : a < b, a \text{ and } b \text{ rational}\} \]
  is a basis that generates a topology different from the lower limit topology on $\mathbb{R}$.
\end{ex}

\begin{proof}
  First we will show that $\mathcal{C}$ is indeed a basis for a topology on
  $\mathbb{R}$. First, for any $x \in \mathbb{R}$, we have $x \in \big[\lfloor x
  \rfloor, \lceil x \rceil + 1\big) \in \mathcal{C}.$ Next suppose that $x \in
  [a_1, b_1) \cap [a_2, b_2)$ for some $a_1,b_1,a_2,b_2 \in \mathbb{Q}$.
  Let $c = \max\{a_1, a_2\}$ and $d = \min\{b_1, b_2\}$. Then $[c, d) \in
  \mathcal{C}$ and
  \[ x \in [c, d) \subset [a_1, b_1) \cap [a_2, b_2). \]
  Therefore $\mathcal{C}$ is indeed a basis for a topology on $\mathbb{R}$.

  Consider the interval $[\pi, 4)$ which is an element of the basis
  $\mathcal{B} = \{[a,b) : a < b, a,b \in \mathbb{R}\}$
  of the lower limit topology on $\mathbb{R}$. Notice $\pi \in [\pi,
  4)$ and there is no $[a, b) \in \mathcal{C}$ such that $\pi \in [a, b) \subset
  [\pi, 4)$. By Lemma 13.3 we know that the topology generated by $\mathcal{C}$
  does not contain the lower limit topology, and thus must be different than the
  lower limit topology.
\end{proof}

\begin{ex}{16.4}
  A map $f: X \to Y$ is said to be an open map if for every open set $U$ of $X$,
  the set $f(U)$ is open in $Y$. Show that $\pi_1: X \times Y \to X$ and $\pi_2:
  X\times Y \to Y$ are open maps.
\end{ex}

\begin{proof}
  Let $W$ be open in $X \times Y$. Then $W$ is a union of basis elements
  \[ W = \bigcup_{\alpha \in J} U_\alpha \times V_\alpha \]
  where $U_\alpha$ and $V_\alpha$ are open sets in $X$ and $Y$ respectively.
  Notice that
  \begin{align*}
    x \times y \in \bigcup_{\alpha \in J} U_\alpha \times V_\alpha
      &\iff x \times y \in U_\alpha \times V_\alpha \text{ for some } \alpha \in J \\
      &\iff x \in U_\alpha \text{ and } y \in V_\alpha \text{ for some } \alpha \in J
        \phantombackup{\bigcup_X}\\
      &\iff x \in \bigcup_{\alpha \in J} U_\alpha \text{ and } y \in \bigcup_{\alpha \in J} V_\alpha \\
      &\iff x \times y \in
        \bigcup_{\alpha \in J} U_\alpha \times \bigcup_{\alpha \in J} V_\alpha
  \end{align*}
  Therefore $W = \cup U_\alpha \times \cup V_\alpha$, so $\pi_1(W) = \cup
  U_\alpha$ and $\pi_2(W) = \cup V_\alpha$. Since each $U_\alpha$ and $V_\alpha$
  is open in $X$ and $Y$ respectively, and these topologies are closed under
  unions, we see that $\pi_1(W)$ and $\pi_2(W)$ are both open. Therefore $\pi_1$
  and $\pi_2$ are open maps.
\end{proof}

\hwpart{2}

\noindent Let $X = \{p \in \mathbb{Z}_+ : p \text{ is prime}\} = \{2, 3, 5, 7, 11, \ldots \}$.
Given $n \in \mathbb{Z}_+$, let $B_n = \{p \in x : p \text{ is not a divisor of
} n\}.$ As an example, $B_{42} = \{5, 11, 13, 17, 19, 23, 29,\ldots\} = X -
\{2, 3, 7\}.$ Let $\mathcal{B} = \{B_n : n \in \mathbb{Z}_+\}.$

\begin{p}{a}
  Prove that $\mathcal{B}$ is a basis for a topology on $X$.
\end{p}

\begin{proof}
  Let $p \in X$. Then $p$ does not divide $1$, so $p \in B_1$. We will show that
  $B_{nm} = B_n \cap B_m$ for any $n, m \in \mathbb{Z}_+$. So let $p \in
  B_{nm}$, which means that $p \nmid nm$.  Then of course $p
  \nmid n$ and $p \nmid m$, so that $p \in B_n \cap B_m$. Conversely if $p \in
  B_n \cap B_m$, which means $p \nmid n$ and $p \nmid m$, since $p$ is prime we
  know that $p \nmid nm$ so that $p \in B_{nm}$. Therefore $B_{nm} = B_n \cap
  B_m$. Of course, this means that for any $p \in B_n \cap B_m$ we have $p \in
  B_{nm} \subset B_n \cap B_m$. Therefore $\mathcal{B}$ is a basis.
\end{proof}

\begin{p}{b}
  The topology generated by $\mathcal{B}$ is a familiar topology. Which one is
  it? Explain.
\end{p}

\begin{solution}
  This is the finite complement topology on $X$. To see why, we can use Lemma
  13.2. For any $B_n \in \mathcal{B}$, $n$ has only finitely many prime
  divisors $p_1,\ldots,p_k$, so that $X - B_n = \{p_1,\ldots,p_k\}$ is finite.
  Thus $\mathcal{B}$ is indeed a collection of open sets in the finite
  complement topology. Next let $U$ be an open set in the finite complement
  topology. Then $X - U$ is either equal to $X$ or finite. If $X - U = X$ then
  $U$ is empty and it is vacuously true that for any $x \in U$ we have $x \in B \subset U$ for
  some $B \in \mathcal{B}$. In the more interesting case, suppose $X - U$ is
  finite. Then $X - U = \{p_1, \ldots, p_k \}$ for some finite number of primes
  $p_1,\ldots,p_k \in X$. The prime divisors of the integer $n = p_1 \cdots
  p_k$ are precisely the primes $p_1,\ldots,p_k$. Thus we have
  \[ B_n = X - \{p_1, \ldots, p_k\} = X - (X - U) = U,\]
  which of course implies that for any $x \in U$ we have $x \in B_n \subset U$.
  Therefore $\mathcal{B}$ is a basis for the finite complement topology.
\end{solution}

\hwpart{3}

\noindent Let $X = \{1, 2, 3, \ldots, n\}.$ Let $\mathcal{T}$ be a topology on $X$.

\begin{p}{a}
  For each $k \in X$, let $B_k$ be the intersection of all elements of $\mathcal{T}$
  which contain $k$. Show that $\mathcal{B} = \{B_k : k \in X\}$ is a basis for
  $\mathcal{T}$.
\end{p}

\begin{proof}
  We will use Lemma 13.2 yet again. First note that $\mathcal{T} \subset
  \mathcal{P}(X)$ is finite, so the collection of open sets containing $k$ is
  finite, and thus each $B_k$ is a finite intersection of open sets. Therefore
  $\mathcal{B}$ is a collection of open sets. Next let $U \in \mathcal{T}$ and
  $x \in U$. Then $x \in B_x$ and since $B_x$ is the intersection of all open
  sets containing $x$, it must be the case that $x \in B_x \subset U$. Therefore
  $\mathcal{B}$ is a basis for $\mathcal{T}$.
\end{proof}

\begin{p}{b}
  Show that any topology on an $n$-point set must have some basis containing $n$
  or fewer open sets. What is the largest possible number of open sets in
  $\mathcal{T}$? (for a fixed $n$)
\end{p}

\begin{proof}
  Any $n$-point set $Y$ is isomorphic to $X$, so let $f: X \to Y$ be a
  bijection. Then it follows from part (a) that the collection $\mathcal{C} = \{ B_{f(k)} : k
  \in X \}$ is a basis for the topology on $Y$. Obviously this collection has
  size $n$ or less (less is possible when $B_j = B_k$ for some $j \ne k$). The
  largest possible number of open sets is the size of the discrete topology
  $|\mathcal{P}(X)| = 2^n$.
\end{proof}

\begin{p}{c}
  Find an infinite space $X$ (and choice of topology) for which the
  process described in part (a) will produce a basis for the topology.
\end{p}

\begin{solution}
  Let $X = \mathbb{Z}_+$ and $\mathcal{T}$ be the order topology. Then we know
  that $\mathcal{T}$ is the discrete topology which contains all singleton sets.
  Thus each $B_k = \{k\}$ and $\mathcal{B}$ is a basis for $\mathcal{T}$.
\end{solution}

\begin{p}{d}
  Find an infinite space $X$ (and choice of topology) for which the
  process described in part (a) will not produce a basis for the topology.
\end{p}

\begin{solution}
  Consider the finite complement from Part 2. When constructing $B_p$, notice
  that for any prime $q \ne p$ we have $p \in X - \{q\} \in \mathcal{T}$.
  Therefore
  \[ B_p \subset \bigcap_{q \ne p} X - \{q\} = X - \bigcup_{q\ne p} \{q\} =
  \{p\}, \]
  and it follows that $B_p = \{p\}$. But $B_p$ is not even open in
  $\mathcal{T}$, since $X - B_p = X - \{p\}$ is neither finite nor equal to $X$.
  Thus $\mathcal{B}$ does not generate $\mathcal{T}$.
\end{solution}

\end{document}
