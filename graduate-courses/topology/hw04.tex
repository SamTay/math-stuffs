\documentclass[11pt]{article}
\usepackage{common}
\usepackage{upgreek}

\begin{document}
\header{4}

% --------------------------------------------------------------
%                         Exercises
% --------------------------------------------------------------

\hwpart{1}

\begin{ex}{16.2}
  If $\mathcal{T}$ and $\mathcal{T}'$ are topologies on $X$ and $\mathcal{T}'$
  is strictly finer than $\mathcal{T}$, what can you say about the corresponding
  subspace topologies on the subset $Y$ of $X$?
\end{ex}

\begin{solution}
  We can say that the subspace topology $\mathcal{T}_Y'$ that $Y$ inherits from
  $\mathcal{T}'$ is finer but possibly not strictly finer than the subspace
  topology $\mathcal{T}_Y$ that $Y$ inherits from $\mathcal{T}$. Suppose $U \in
  \mathcal{T}_Y$. Then $U = Y \cap V$ for some $V \in \mathcal{T}$. Since
  $\mathcal{T} \subset \mathcal{T}'$, we have $V \in \mathcal{T}'$ so that $U =
  Y \cap V \in \mathcal{T}_Y'$ as well. Therefore $\mathcal{T}_Y \subset
  \mathcal{T}_Y'$.

  For an example where the inherited subspace topologies are equal, consider
  when $\mathcal{T} = \mathbb{R}, \mathcal{T}' = \mathbb{R}_K,$ and $Y = (-2,
  -1).$ Let $\mathcal{B}_K$ be the standard basis for $\mathbb{R}_K$ and
  $\mathcal{B}$ be the standard basis for $\mathbb{R}$. Next let $B_K
  \cap Y$ be any basis element of the inherited subspace topology
  $\mathcal{T}_Y'$, as given in Lemma 16.1, where $B_K \in \mathcal{B}_K.$
  Notice that due to our choice of $Y$ there is always an interval $B \in
  \mathcal{B}$ such that $B_K \cap Y = B \cap Y$. Thus by Lemma 13.3 we have
  $\mathcal{T}' \subset \mathcal{T}.$ So in this case, $\mathcal{T}'$ is not
  strictly finer than $\mathcal{T}$.
\end{solution}

\begin{ex}{16.8}
  If $L$ is a straight line in the plane, describe the topology $L$ inherits as
  a subspace of $\mathbb{R}_\ell \times \mathbb{R}$ and as a subspace of
  \(\mathbb{R}_\ell \times \mathbb{R}_\ell\). In each case it is a familiar
  topology.
\end{ex}

\begin{solution}
  The topology $L$ inherits depends on its slope. When inherited from
  $\mathbb{R}_\ell \times \mathbb{R}$:

  \vspace{3in}

  $m = \infty$
    \hfill $m = 0$
    \hfill $m > 0$
    \hfill $m < 0$

  $L \longleftrightarrow \mathbb{R}$
    \hfill $L \longleftrightarrow \mathbb{R}_\ell$
    \hfill $L \longleftrightarrow \mathbb{R}_\ell$
    \hfill $L \longleftrightarrow \mathbb{R}_\ell$

  \newpage

  When inherited from $\mathbb{R}_\ell \times \mathbb{R}_\ell$:

  \vspace{3in}
  $m = \infty$ \phantombackup{hi th}
    \hfill $m = 0$ \phantombackup{hi }
    \hfill $m > 0$ \phantombackup{hi the}
    \hfill $m < 0$

  $L \longleftrightarrow \mathbb{R}_\ell$
    \hfill $L \longleftrightarrow \mathbb{R}_\ell$
    \hfill $L \longleftrightarrow \mathbb{R}_\ell$
    \hfill $L \longleftrightarrow \mathcal{P}(\mathbb{R})$

  Note that since the lower limit and upper limit topologies are isomorphic,
  these $\mathbb{R}_\ell$'s could also be considered naturally as
  $\mathbb{R}_u$'s. For example, it would be reasonable to say that the topology
  $L$ inherits as a subspace of $\mathbb{R}_\ell \times \mathbb{R}$ when its
  slope is positive is isomorphic to $\mathbb{R}_u$ if you consider the order of
  the points on $L$ to increase as its $y$-coordinate increases.
  But the answers above are consistent if you treat the order
  topology on $L$ as inheriting the dictionary order from $\mathbb{R} \times
  \mathbb{R}$, that is, defining $(x_1, mx_1 + b) < (x_2, mx_2 + b)$ as long as $x_1 <
  x_2$ for points on the line $L = \{(x, mx + b) : x \in \mathbb{R}\}$.
\end{solution}


\begin{ex}{16.10}
  Let $I = [0, 1]$. Compare the product topology on $I \times I$, the dictionary
  order topology on $I \times I$, and the topology $I \times I$ inherits as a
  subspace of $\mathbb{R} \times \mathbb{R}$ in the dictionary order topology.
\end{ex}

\begin{solution}
  I assume the product topology is referring to the product of each $I$ under
  the order topology. Let us denote
  \begin{align*}
    \mathcal{T} \;&= \text{ the order topology on } I \\
    \mathcal{T}_\times &= \text{ the product topology on } I \times I \\
    \mathcal{T}_< &= \text{ the dictionary order topology on } I \times I \\
    \mathcal{T}_s \;&= \text{ the subspace topology on } I \times I \\
    \mathcal{B} \;\;&= \text{ the usual basis for the order topology } \mathcal{T} \\
    \mathcal{B}_\times &= \{U \times V : U, V \in \mathcal{T} \} \\
    \mathcal{B}_< &= \text{ the usual basis for the dictionary order topology on } I \times I \\
    \mathcal{B}_s \;&= \{B \cap I \times I : B \in \mathcal{C} \} \\
    \mathcal{C} \;\;&= \text{ the usual basis for the dictionary order topology on }
      \mathbb{R} \times \mathbb{R}
  \end{align*}

  I argue that $\mathcal{T}_s$ is finer than both $\mathcal{T}_\times$ and
  $\mathcal{T}_<$, but that $\mathcal{T}_\times$ and $\mathcal{T}_<$ are
  incomparable with each other.
  First, I'll show incomparability using Lemma 13.3.
  Let
  \[ B_\times = \Big(\frac{1}{4}, \frac{1}{2}\Big) \times \Big[0, \frac{1}{2}\Big)
    \text{ and } x = \frac{1}{3} \times 0.
  \]
  Since $(\frac{1}{4}, \frac{1}{2})$ and $[0, \frac{1}{2})$ are both open in
  $\mathcal{T}$, we have $B_\times  \in \mathcal{B}_\times.$
  However, notice that $x \in B_\times$ but there is no $B \in \mathcal{B}_<$
  such that $x \in B \subset B_\times.$ This is easy to see
  geometrically: any basis element $B \in \mathcal{B}_<$ that contains
  $x$ will have to spill over and contain an element $a \times 1$ where
  $a < \frac{1}{3}$, but all the elements of $B_\times$ have
  $y$-coordinate less than $\frac{1}{2}$. By Lemma 13.3 we know that
  $\mathcal{T}_\times \not\subset \mathcal{T}_<$. Next let
  \[ B_< = \Big[0 \times 0, 0 \times \frac{1}{4}\Big)
    \text{ and } x = 0 \times 0.
  \]
  Notice that $B_< \in \mathcal{B}_<$ and $x \in B_<$. However, if we try to
  find subsets $U, V$ such that $x \in U \times V \subset B_<$, this will force
  $U = \{0\}.$ Of course, $\{0\}$ is not open in $\mathcal{T}$, and therefore
  there is no basis element $B \in \mathcal{B}_\times$ such that $x \in B
  \subset B_<$. By Lemma 13.3, $\mathcal{T}_< \not\subset \mathcal{T}_\times$.
  Therefore $\mathcal{T}_\times$ and $\mathcal{T}_<$ are incomparable.

  Next we show that $\mathcal{T}_\times \subset \mathcal{T}_s$; to this end, let
  $B_\times \in \mathcal{B}_\times$ and $x \in B_\times$. Then $B_\times = U \times V$ for some open
  sets $U, V \in \mathcal{T}$ and $x = u \times v$ for some $u \in U$ and $v \in
  V$. Now there exists $B_v \in \mathcal{B}$ such that $v \in B_v \subset V$. We
  can construct an interval $(a,b) \subset \mathbb{R}$ such that $v \in (a,
  b) \cap I \subset B_v$; specifically, if $B_v = [0, d)$ then choose
  $(a, b) = (-1, d)$, if $B_v = (c, d)$ then choose $(a, b) = (c, d)$, and if
  $B_v = (c, 1]$ then choose $(a, b) = (c, 2)$. Now $v \in (a, b) \cap I
  \subset B_v$. Finally, let
  \begin{align*}
    B_s &= (u \times a, u \times b) \cap (I \times I).\\
        &= \{u\} \times ((a, b) \cap I) \\
        &\subset B_u \times B_v \\
        &\subset U \times V = B_\times
  \end{align*}
  Now it is clear that $B_s \in \mathcal{B}_s$ and $x \in B_s \subset B_\times$.
  Therefore by Lemma 13.3 $\mathcal{T}_\times \subset \mathcal{T}_s$.

  Finally, we'll show that $\mathcal{T}_< \subset \mathcal{T}_s$. Unfortunately, I've only just
  realized that Lemma 13.3 is a bit of a roundabout way to show containment.
  Instead, note that if $\mathcal{T}_1, \mathcal{T}_2$ are topologies on a space
  $X$ and $\mathcal{S}$ is a subbasis for $\mathcal{T}_1$, then $\mathcal{S}
  \subset \mathcal{T}_2$ implies $\mathcal{T}_1 \subset \mathcal{T}_2$. This is
  because $T_1$ is made up of arbitrary unions of finite intersections of
  elements of $\mathcal{S}$, and $\mathcal{T}_2$ is closed under arbitrary
  unions and finite intersections; it's rather obvious in hindsight. We know
  that the open rays $[0 \times
  0, a \times b)$ and $(a \times b, 1 \times 1]$ form a subbasis for
  $\mathcal{T}_<$. It is easy to see that
  \begin{align*}
    [0 \times 0, a \times b) &= (-\infty, a \times b) \;\cap\; I \times I \\
    [a \times b, 1 \times 1) &= (a \times b, \infty) \;\cap\; I \times I
  \end{align*}
  It is now clear that these open rays are open in the subspace topology, and
  as reasoned above, we conclude $\mathcal{T}_< \subset \mathcal{T}_s$.

\end{solution}

\begin{ex}{17.2}
  Show that if $A$ is closed in $Y$ and $Y$ is closed in $X$, then $A$ is closed
  in $X$.
\end{ex}

\begin{proof}
  Let $A$ be closed in $Y$ and $Y$ be closed in $X$. Then by Theorem 17.2 $A = Y
  \cap C$  for some closed set $C$ in $X$. Notice that $A$ is an intersection of two
  closed sets in $X$, so $A$ is also closed in $X$.
\end{proof}

\begin{ex}{17.3}
  Show that if $A$ is closed in $X$ and $B$ is closed in $Y$, then $A \times B$
  is closed in $X \times Y$.
\end{ex}

\begin{proof}
  Suppose $A$ is closed in $X$ and $B$ is closed in $Y$, which means that $X-A$
  is open in $X$ and $Y - B$ is open in $Y$. Then $(X-A) \times Y$ and $X \times
  (Y-B)$ are both open in $X \times Y$, so their union
  \[ (X - A) \times Y \;\cup\; X \times (Y-B) \;=\; X \times Y - A \times B \]
  is open in $X \times Y$. Therefore $A \times B$ is closed in $X \times Y$.
\end{proof}

\begin{ex}{17.4}
  Show that if $U$ is open in $X$ and $A$ is closed in $X$, then $U - A$ is open
  in $X$, and $A - U$ is closed in $X$.
\end{ex}

\begin{proof}
  Suppose $U$ is open in $X$ and $A$ is closed in $X$. Then $X - U$ is closed
  and $X - A$ is open. Notice that $U - A = U \cap (X - A)$ is an intersection
  of open sets and $A - U = A \cap (X - U)$ is an intersection of closed sets,
  so $U - A$ is open and $A - U$ is closed.
\end{proof}

\begin{ex}{17.6}
  Let $A, B,$ and $A_\alpha$ denote subsets of a space $X$. Prove the following:
\end{ex}

\begin{p}{a}
  If $A \subset B$, then $\cl{A} \subset \cl{B}.$
\end{p}

\begin{proof}
  Suppose $A \subset B$ and let $x \in \cl A$. By Theorem 17.5, all
  neighborhoods of $x$ intersect $A$, and since $A \subset B$, these
  neighborhoods intersect $B$ as well. By the same theorem, $x \in \cl B$. Thus
  $\cl A \subset \cl B$.
\end{proof}

\begin{p}{b}
  \(\cl{A \cup B} = \cl{A} \cup \cl{B}\).
\end{p}

\begin{proof}
  $(\subset)$ Suppose $x \not\in \cl A \cup \cl B$. Then
  \[ x \in X - (\cl A \cup \cl B) = (X - \cl A) \cap (X - \cl
  B),\]
  where $(X - \cl A) \cap (X - \cl B)$ is an open set containing $x$
  disjoint from $A \cup B$. Thus by Theorem 17.5, $x \not\in \cl{A \cup
  B}$.

  $(\supset)$ Suppose $x \in \cl A \cup \cl B$. Then all neighborhoods
  of $x$ intersect $A$ or all neighborhoods of $x$ intersect $B$. Then of
  course, all neighborhoods of $x$ intersect $A \cup B$, so that $x \in \cl{A \cup
  B}$.
\end{proof}

\begin{p}{c}
  \(\cl{\bigcup A_\alpha} \supset \bigcup \cl{A_\alpha}\); give an
  example where equality fails.
\end{p}

\begin{proof}
  Let $x \in \bigcup \cl A_\alpha$, so that $x \in \cl A_\alpha$ for
  some $\alpha$. Then all neighborhoods of $x$ intersect $A_\alpha$, and hence
  intersect $\bigcup A_\alpha$. Thus $x \in \cl{\bigcup A_\alpha}$ and
  therefore
  $\bigcup \cl A_\alpha \subset \cl{\bigcup A_\alpha}$.

  For an example where equality fails, consider the collection of singleton sets
  $\{\frac{1}{n}\}_{n \in \mathbb{Z}_+}$ in the context of the order topology on
  $\mathbb{R}$. For each $n$, $\cl{\{\frac{1}{n}\}} = \{\frac{1}{n}\}$ since
  singleton sets are closed, so $\bigcup \cl{\{\frac{1}{n}\}} = \bigcup
  \{\frac{1}{n}\}$. However as shown in class, $\cl{\bigcup
  \{\frac{1}{n}\}} = \{0\} \cup \bigcup \{\frac{1}{n}\}.$
\end{proof}

\hwpart{2}

\noindent Show that if $X$ is an ordered set for which the order topology is the discrete
topology, then every element of $X$ except the largest one (if there is a largest
one) has an immediate successor. Through a practically identical proof, it's also
possible to show that every element except the smallest one (if there is a
smallest one) has an immediate predecessor.

\begin{proof}
  We proceed by contrapositive. Suppose there exists $x \in X$ such that $x$ is
  not the largest element and $x$ does not have an immediate successor. We will
  show that $(x, \infty)$ is not closed. Let $U$
  be any neighborhood of $x$. Then there is a basis interval $B$
  such that $x \in B \subset U$. We need to find an element $c$ in the
  intersection of $U$ and $(x, \infty)$. There are a few cases for the shape of $B$:
  \begin{description}
    \item[Case 1: $B = (a, b)$.]
      In this case, $x < b$ and since $x$ does not have an immediate successor,
      there must exist $c$ such that $x < c < b$. Now $c \in (a, b) \cap (x, \infty)$.
    \item[{Case 2: $B = [a, b)$.}]
      Just as in Case 1, since $x < b$ with no immediate successor, we choose
      $c$ such that $x < c < b$. Now $c \in [a, b) \cap (x, \infty)$.
    \item[{Case 3: $B = (a, b]$.}]
      In this case, recall $x$ is assumed not the largest element, so $x < b$ and
      we can choose $c = b$. Then $c \in (a, b] = (a, b] \cap (x, \infty).$
  \end{description}
  In all cases we have shown that $B$ (and hence $U$) intersects $(x, \infty)$ at a point $c
  \neq x$. This shows that $x$ is a limit point of $(x, \infty)$, however
  clearly $x \not\in (x, \infty)$. By Corollary 17.7, $(x, \infty)$ is not
  closed, so this is not the discrete topology.
\end{proof}

\end{document}
