\documentclass[11pt]{article}
\usepackage{common}

\begin{document}
\header{10}

% --------------------------------------------------------------
%                         Exercises
% --------------------------------------------------------------

\hwpart{1}

\begin{ex}{26.1(a)}
  Let $\mathcal{T}$ and $\mathcal{T}'$ be two topologies on the set $X$; suppose
  that $\mathcal{T} \subset \mathcal{T}'$. What does compactness of $X$ under
  one of these topologies imply about compactness under the other?
\end{ex}

\begin{solution}
  Compactness in $\mathcal{T}$ does not guarantee compactness in $\mathcal{T}'$;
  consider the interval $[0, 1]$ closed as a subspace of the usual topology on
  $\mathbb{R}$, however $[0, 1]$ is obviously not closed in the discrete
  topology.

  On the other hand, compactness in $\mathcal{T}'$ does guarantee compactness in
  $\mathcal{T}$, since a covering $\mathcal{A}$ of sets open in $\mathcal{T}$
  are also open in $\mathcal{T}'$, hence a finite subcover exists in
  $\mathcal{T}$ as well.
\end{solution}

\begin{ex}{26.1(b)}
  Show that if $X$ is compact Hausdorff under both $\mathcal{T}$ and
  $\mathcal{T}'$, then either $\mathcal{T}$ and $\mathcal{T}'$ are equal or they
  are not comparable.
\end{ex}

\begin{proof}
  Suppose for contradiction that one topology strictly contains the other,
  without loss of generality, $\mathcal{T} \subsetneq \mathcal{T}'$. Let $id: X
  \to X$ be the identity map from $\mathcal{T}'$ into $\mathcal{T}$. Then $id$
  is bijective and for any $U \in \mathcal{T}$, $id^{-1}(U) = U \in
  \mathcal{T}'$, so $id$ is continuous as well. By Theorem 26.6, $id$ is a
  homeomorphism. But this implies that for any $V \in \mathcal{T}'$, $id(V) = V
  \in \mathcal{T}$ as well, contradicting our assumption that $\mathcal{T}
  \subsetneq \mathcal{T}'$. Thus one topology cannot strictly contain the other,
  so it must be the case that either $\mathcal{T} = \mathcal{T}'$ or they are
  not comparable.
\end{proof}

\begin{ex}{26.3}
  Show that a finite union of compact subspaces of $X$ is compact.
\end{ex}

\begin{proof}
  Let $Y_1,\ldots,Y_n$ be a finite number of compact subspaces of $X$. Let
  $\mathcal{A}$ be a covering of the union $\cup_{i=1}^n Y_i$ of sets open in
  $X$. Then $\mathcal{A}$ obviously covers each subspace $Y_i$, and by Lemma
  26.1 there is a finite subcollection $\mathcal{A}_i \subset \mathcal{A}$ that
  covers $Y_i$, for each $i=1,\ldots,n$. Then $\cup_{i=1}^n \mathcal{A}_i$ is a
  finite subcollection of $\mathcal{A}$ that covers $\cup_{i=1}^n Y_i$;
  again by Lemma 26.1, $\cup_{i=1}^n Y_i$ is compact.
\end{proof}

\begin{ex}{26.7}
  Show that if $Y$ is compact, then the projection $\pi_1 : X \times Y \to X$ is
  a closed map.
\end{ex}

\begin{proof}
  Let $U$ be closed in $X \times Y$. Then we wish to show that $\pi_1(U)$ is
  closed in $X$, so pick $x_0 \in X - \pi_1(U)$; we will find an open set $W$
  such that $x_0 \in W \subset X - \pi_1(U)$. Notice that $(X \times Y) - U$ is
  open and contains $\{x_0\} \times Y$. By the Tube Lemma, there is a
  neighborhood $W$ of $x_0$ such that
  \[ \{x_0\} \times Y \subset W \times Y \subset (X \times Y) - U. \]
  Thus, for any $y \in Y$, there is no $x \in W$ such that $x \times y \in U$.
  Hence $x_0 \in W \subset X - \pi_1(U)$, so $X - \pi_1(U)$ is open and
  $\pi_1(U)$ is closed.
\end{proof}

\begin{ex}{26.8}
  Let $f: X \to Y$; let $Y$ be compact Hausdorff. Then $f$ is continuous if and
  only if the \emph{graph} of $f$,
  \[ G_f = \{x \times f(x) : x \in X\},\]
  is closed in $X \times Y$.
  % Hint: if G_f is closed and V is a neighborhood of f(x_0), then the
  % intersection of G_f and X \times (Y - V) is closed. Apply Exercise 7.
\end{ex}

\begin{proof}
  $(\Longrightarrow)$ Suppose $f$ is continuous. Let $x \times y \in (X \times
  Y) - G_f$. Then $y \neq f(x)$ and since $Y$ is Hausdorff we can find disjoint open
  neighborhoods $U$ of $y$ and $V$ of $f(x)$. Then $x \in f^{-1}(V)$ so
  $x \times y$ is in $f^{-1}(V) \times U$, which is open because $f$ is
  continuous. Furthermore, for any $z \in f^{-1}(V),$ $f(z) \in V$ and thus
  $f(z) \not\in U$, so $f^{-1}(V) \times U \subset (X \times Y) - G_f.$
  Therefore $G_f$ is closed.

  $(\Longleftarrow)$ Suppose $G_f$ is closed. To show $f$ is continuous, let $U$
  be open in $Y$. Then $X \times U$ is open, so the complement
  \[(X \times Y) - (X \times U) = X \times (Y - U) \]
  is closed. Then the intersection
  \[ (X \times (Y-U)) \cap G_f = \{x \times f(x): f(x) \not\in U\} \]
  is also closed. Since $Y$ is compact we know from Exercise 26.7 that $\pi_1$ is a
  closed map, so
  \[ \pi_1(\{x \times f(x): f(x) \not\in U\}) = X - f^{-1}(U) \]
  is closed in $X$. Hence $f^{-1}(U)$ is open and $f$ is continuous.

\end{proof}

\begin{ex}{27.1}
  Prove that if $X$ is an ordered set in which every closed interval is compact,
  then $X$ has the least upper bound property.
\end{ex}

\begin{proof}
  We proceed by contrapositive. Let $A \subset X$ be a nonempty set that is
  bounded above without a least upper bound. Let $B$ be the (nonempty) set of
  upper bounds for $A$. A couple things to
  note initially: $A$ cannot have a maximum element, as it would be the least
  upper bound; $B$ cannot have a least element, as it would be the least upper
  bound; $A$ and $B$ are disjoint, otherwise they would intersect at the
  least upper bound.

  Let $a_0 \in A$ and $b_0 \in B$. Since $A$ has no maximum and $B$ has no
  minimum, we can find $c \in A$ and $d \in B$ such that $a_0 < c$ and
  $d < b_0$. We will show that
  \[ \mathcal{A}
        = \{(a_0, a): a \in A \text{ and } a > a_0\}
          \cup \{(b, b_0): b \in B \text{ and } b < b_0\}
        \]
  is a covering of $[c, d]$.

  Let $x \in [c, d]$. If $x \in B$ then there must exist $b \in B$ with $b < x$ so
  that $x \in (b, d] \subset (b, b_0).$ If $x \not\in B$ then $x$ is not
  an upper bound for $A$, so there exists an $a > x$ so that $x \in [c, a)
  \subset (a_0, a).$ Therefore $\mathcal{A}$ covers $[c, d]$.

  Notice $\mathcal{A}$ is a union of two sets which each consist of nested intervals.
  Any finite subcollection will have the form
  \begin{align*}
    \mathcal{C} &= \bigcup_{i=1}^n \{(a_0, a_i)\} \cup \bigcup_{j=1}^m \{(b_j,
    b_0)\}
  \end{align*}
  and
  \begin{align*}
    \bigcup \mathcal{C} &= (a_0, a_N) \cup (b_M, b_0)
  \end{align*}
  where $a_N = \max\{a_i\}$ and $b_M = \min\{b_j\}$. Then $a_N, b_M \in [c, d]$,
  but $a_N, b_M \not\in \bigcup \mathcal{C}$, as we stated
  earlier that $A$ and $B$ are disjoint. Thus no finite subcollection of
  $\mathcal{A}$ can still cover $[c, d]$, and we have found a closed interval
  that is not compact.
\end{proof}

\hwpart{2}

Let $C$ be the Cantor set. % p178 for definition
Also, give $\{0, 1\}$ the discrete topology and $\{0, 1\}^\omega$ the product
topology. Note $\{0, 1\}^\omega$ can be regarded as the space of all sequences
of zeroes and ones.

Munkres defines the Cantor set by constructing a sequence of sets $A_n$ each of
which is a union of disjoint closed intervals of length $\frac{1}{3^n}$. Define
a function $f: C \to \{0, 1\}^\omega$ by saying that the $n^{\text{th}}$ term in
the sequence $f(x)$ is $0$ if for any $k \in \mathbb{Z}$, $x \in
\qty[\frac{3k}{3^n}, \frac{1 + 3k}{3^n}] \subset A_n$ and the $n^{\text{th}}$ term
of $f(x)$ is $1$ if for any $k \in \mathbb{Z}$, $x \in \qty[\frac{2 + 3k}{3^n}, \frac{3
+ 3k}{3^n}] \subset A_n$.

The way we construct the Cantor set is by removing the middle third of an
interval, then removing the middle third of the remaining left and right
intervals, and so on. What $f$ does is record, at each step of the iteration,
whether $x$ was in an interval just left of or just right of a middle third that
was removed. For example, for $x = \frac{7}{27}$, when we remove the middle
third from $[0, 1]$, $\frac{7}{27}$ lies in the left third $[0, \frac{1}{3}]$,
and then when we remove the middle thid from $[0, \frac{1}{3}]$, $\frac{7}{27}$
lies in the right third $[\frac{2}{9}, \frac{1}{3}]$, and then when we remove
the middle third from $[\frac{2}{9}, \frac{1}{3}]$, $\frac{7}{27}$ lies in the
left third $[\frac{2}{9}, \frac{7}{27}]$, and then after that, it's always in
the right third. Thus, ``directions" for finding $\frac{7}{27}$ within the
Cantor set are left, right, left, right, right, right, $\ldots$, which gets
represented as $f\qty(\frac{7}{27}) = (0, 1, 0, 1, 1, 1, \ldots).$

\begin{p}{a}
  Show that $C$ is compact Hausdorff.
\end{p}

\begin{proof}
  Recall that closed subspaces of compact spaces are compact, and subspaces of
  Hausdorff spaces are Hausdorff; since $C \subset [0, 1]$ and $[0, 1]$ is
  compact Hausdorff, it suffices to show that $C$ is closed. But $C$ is defined
  as an intersection $\bigcap_{n \in \mathbb{Z}_+} A_n$ where each $A_n$ is a
  finite union of closed intervals; hence $C$ is closed. Therefore $C$ is
  compact Hausdorff.
\end{proof}

\begin{p}{b}
  Show that $f$ is bijective.
\end{p}

\begin{proof}
  First we'll show that $f$ is surjective. Let $\vb x \in \{0, 1\}^\omega$.
  Define
  \begin{align*}
    B_1 &=
    \begin{cases}
      \qty[0, \frac{1}{3}] &\text{ if } x_1 = 0 \\
      \qty[\frac{2}{3}, 1] &\text{ if } x_1 = 1
    \end{cases} \\
    B_n &= B_{n-1} \cap
    \begin{cases}
      \bigcup_{k = 0}^\infty \qty[\frac{3k}{3^n}, \frac{1 + 3k}{3^n}] &\text{ if } x_n = 0 \\
      \bigcup_{k = 0}^\infty \qty[\frac{2 + 3k}{3^n}, \frac{3 + 3k}{3^n}] &\text{ if } x_n = 1
    \end{cases}
  \end{align*}
  where each $B_i$ is defined as interval subsets of $C$. By construction
  $\mathcal{B} = \{B_n\}_{n=1}^\infty$ has the finite
  intersection property, and since $C$ is compact, we know that $B =
  \bigcap_{n=1}^\infty B_n$ is nonempty and for $b \in B$, $f(b) = \vb x$.
  Therefore $f$ is surjective.

  Next let $a, b \in C$ such that $a \neq b$. Let $n$ be the least positive
  integer so that $\frac{1}{3^n} < \abs{a - b}$ Then $a, b$ end up in different
  thirds at the $n^{\text{th}}$ iteration, such that $\pi_n(f(a)) \neq
  \pi_n(f(b))$, hence $f(a) \neq f(b)$. Therefore $f$ is injective.
\end{proof}

\begin{p}{c}
  Show that $f$ is continuous.
\end{p}

\begin{proof}
  Let $c \in C$ and $V$ be a basis neighborhood of $f(c)$. Then $V = \prod_{n \in
  \mathbb{Z}_+} V_n$ where only finitely many $V_n \neq \{0, 1\}$. So there
  exists $N$ such that $V_n = \{0, 1\}$ for all $n > N$. Then we can define
  \[ U = \prod_{n \in \mathbb{Z}_+}
    \begin{cases}
      \{\pi_n(f(c))\} &\text{ if } n \leq N \\
      \{0, 1\} &\text{ if } n > N
    \end{cases}
  \]
  so that $f(c) \in U \subset V$. We want to find a neighborhood of $c$ whose
  image is contained in $U$; this amounts to making sure that the neighborhood
  agrees with $c$ on the directions of the first $N$ iterations of the recursive
  Cantor definition. We can do this by choosing a suitably small open ball
  around $c$, namely $B(c, \frac{1}{3^N})$. For any $x \in B(c, \frac{1}{3^N})$ the
  distance between $x$ and $c$ is less than $\frac{1}{3^N}$ so for $n \leq N$,
  $x$ and $c$ fall into the same third of the interval of length $\frac{1}{3^n}$. Thus
  $f\qty(B(c, \frac{1}{3^N})) \subset U \subset V$, and $f$ is continuous.
\end{proof}

\begin{p}{d}
  Show that $f$ is a homeomorphism.
\end{p}

\begin{proof}
  Since $\{0, 1\}$ is discrete, it is Hausdorff, and we know that $\{0,
  1\}^\omega$ is Hausdorff as well by Theorem 19.4. We showed that $C$ is
  compact in part (a) and that $f$ is bijective and continuous in parts (b) and
  (c). By Theorem 26.6, $f$ is a homeomorphism.
\end{proof}


\end{document}
