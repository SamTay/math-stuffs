
\documentclass{article}
\newenvironment{problem2}[1]{\noindent {\bf (#1}}
{\medskip}

\newenvironment{problem1}[1]{\noindent {\bf Problem #1:}}
{\medskip}
\usepackage{graphicx}
\usepackage{amsmath,amssymb,amsthm,amsfonts,graphicx,url,colordvi}
\usepackage{amssymb}
\usepackage{mathtools}
\makeatletter
\def\imod#1{\allowbreak\mkern10mu({\operator@font mod}\,\,#1)}
\makeatother

\usepackage{hyperref}
\usepackage{amssymb}
%\usepackage{fullpage}
\hypersetup{pdfborder={0 0 0}}
\DeclareMathOperator{\lcm}{lcm}

\title{Math 327: Final Take-home Solutions}
\author{by Sam Tay}
\date{\today}

\begin{document}
%\maketitle
\begin{flushright}Sam Tay\\ Professor Milnikiel \\ Math 335 \\ Section 11: 2, 8, 16, 25, 29, 50, 51\\ 12/23/11
\end{flushright}

\begin{problem1}{2} \\ We have $$\mathbb{Z}_3 \times \mathbb{Z}_4=\{(0,0),(0,1),(0,2),(0,3), (1,0), (1,1), (1,2), (1,3), (2,0),(2,1),(2,2),(2,3)\}$$ and this group is cyclic because $\gcd(3,4)=1$ and we see that $\langle (1,1) \rangle = \mathbb{Z}_3 \times \mathbb{Z}_4$.\\

\end{problem1}


\begin{problem1}{8}\\The cyclic subgroups of $\mathbb{Z}_6 \times \mathbb{Z}_8$ are generated by some $(a_1,a_2)$ where $a_1\in\mathbb{Z}_6$ and $ a_2\in\mathbb{Z}_8$. The order $|(a_1,a_2)|=\lcm(r_1,r_2)$ where $r_1=|a_1|$ in $\mathbb{Z}_6$ and $r_2=|a_2|$ in $\mathbb{Z}_8$. By the Theorem of Lagrange, we know that the orders $r_1 \vert 6$ and $r_2\vert 8$. Clearly then, their least common multiple is maximized when $r_1=6$ and $r_2=8$, which includes all factors of $6$ and $8$. Thus the largest order of an element in $\mathbb{Z}_6 \times \mathbb{Z}_8$ is $\lcm(6,8)=24$.

With similar reasoning, the largest order of $\mathbb{Z}_{12} \times \mathbb{Z}_{15}$ is $\lcm(12,15)=60.$\\
\end{problem1}

\begin{problem1}{16}\\ Yes, the groups $\mathbb{Z}_2 \times \mathbb{Z}_{12}$ and $\mathbb{Z}_4 \times \mathbb{Z}_{6}$ are isomorphic. From Theorem 11.5 we know that since $\gcd(2,3)=1$ and $\gcd(3,4)=1$, $$\mathbb{Z}_{6}\cong\mathbb{Z}_2 \times \mathbb{Z}_3 \qquad \text{ and } \qquad \mathbb{Z}_{12}\cong\mathbb{Z}_3 \times \mathbb{Z}_4,$$ so we have $$\mathbb{Z}_4 \times \mathbb{Z}_{6}\ \cong\ \mathbb{Z}_4 \times \mathbb{Z}_2 \times \mathbb{Z}_3\ \cong \ \mathbb{Z}_2 \times \mathbb{Z}_{12}.$$

\end{problem1}


\begin{problem1}{25}\\
Note that $1089=3^2\cdot 11^2$. Keeping Theorem 11.5 in mind, we see that there are two ways to combine each factor 3 and 11 such that there are $2\cdot2=4$ abelian groups of order 1089, up to isomorphism. The possible abelian groups are
\begin{enumerate}
\item{$\mathbb{Z}_3 \times \mathbb{Z}_3 \times \mathbb{Z}_{11} \times \mathbb{Z}_{11}$}
\item{$\mathbb{Z}_9 \times \mathbb{Z}_{11} \times \mathbb{Z}_{11}$}
\item{$\mathbb{Z}_3 \times \mathbb{Z}_3 \times \mathbb{Z}_{121}$}
\item{$\mathbb{Z}_9 \times \mathbb{Z}_{121}$} 
\end{enumerate}

\end{problem1}



\begin{problem1}{29 (a) }\\ Let $p$ be a prime number. By Theorem 11.5, we can infer that the number of abelian groups, up to isomorphism, of order $p^n$ is exactly the number of partitions $p(n)$ of $n$ elements. This is summarized in the table below.

\begin{center}
\begin{tabular}{ l | c | c  | c | c| c| c| c| r }
\phantom{111111111111111}$n$ & 2 & 3 & 4 & 5 & 6 & 7 & 8\\\hline
number of groups & 2 & 3 & 5 & 7 & 11 & 15 & 22 \\
\end{tabular}
\end{center}

\end{problem1}

\begin{problem2}{b)}  Referring to the table above, if $p,q,$ and $r$ are distinct primes then we can combine the factors such that if a group $G$ has order $p^nq^mr^l$, there are exactly $p(n)p(m)p(l)$ possible structures up to isomorphism.
\end{problem2}

\begin{problem2}{i)} If an abelian group has order $p^3q^4r^7$, then there are $3\cdot5\cdot15=225$ possibilities.
\end{problem2}

\begin{problem2}{ii)} If an abelian group has order $(qr)^7=q^7r^7$, then there are $15\cdot15=225$ possibilities.
\end{problem2}

\begin{problem2}{iii)} If an abelian group has order $q^5r^4q^3=q^8r^4$, then there are $22\cdot5=110$ possibilities.\\
\end{problem2}


\begin{problem1}{50} \\ Let $H$ and $K$ be groups and let $G=H\times K$. Let $e_H,e_K$ be the identities of $H, K$ respectively, and let us denote the subgroups $H\times\{e_K\}\le G$ and $\{e_H\}\times K\le G$ by $H_G$ and $K_G$ respectively. \\\end{problem1}



\begin{problem2}{a)} Every element of $G$ is of the form $hk$ for some $h\in H_G$ and $k\in K_G$.
\end{problem2}

\begin{proof} Suppose $x\in G$. Then $x\in H\times K$ which means $x=(h',k')$ for $h'\in H$ and $k'\in K$. Clearly since $h'=h'e_H$ and $k'=e_Kk'$, we have $x=(h'e_H, e_Kk')=(h',e_K)(e_H,k')$ where $(h',e_K)\in H_G$ and $(e_H,k')\in K_G$. Since $x$ was arbitrary, we conclude that all elements of $G$ are of the form $hk$ for some $h\in H_G$ and $k\in K_G$.
\end{proof}

\begin{problem2}{b)} $hk=kh$ for all $h\in H_G$ and $k\in K_G$.
\end{problem2}

\begin{proof} Let $h\in H_G$ and $k\in K_G$. Then $h=(h',e_K)$ and $k=(e_H,k')$ for some $h'\in H$ and $k'\in K$. Since the identity element of a group commutes with all elements of the group, $$hk=(h',e_K)(e_H,k')=(h'e_H,e_Kk')=(e_Hh', k'e_K)=(e_H,k')(h',e_K)=kh.$$ \end{proof}

\begin{problem2}{c)} $H_G \cap K_G = \{(e_H,e_K)\}$.
\end{problem2}

\begin{proof} Suppose that $x\in H_G \cap K_G $. Then $x\in H_G$ and $x\in K_G$, which means $x=(h,e_K)$ and $x=(e_H,k)$ for some $h\in H$ and $k\in K$. It follows immediately that $h=e_H$ and $k=e_K$, such that $x=(e_H,e_K)$. Therefore this is the only element in the intersection, such that $H_G \cap K_G = \{(e_H,e_K)\}$.
\end{proof}

\begin{problem1}{51}\\ Suppose $H$ and $K$ are subgroups of $G$ satisfying the properties \textbf{(a)}, \textbf{(b)}, and \textbf{(c)} above. Then for every $g\in G$, the expression $g=hk$ for $h\in H$ and $k\in K$ is unique. Further, if we rename $g$ as $(h,k)$ then $G$ is isomorphic to $H\times K$.
\end{problem1}

\begin{proof} First, we know that if property \textbf{(a)} is satisfied then every element $g\in G$ is of the form $hk$ for some $h\in H$ and $k\in K$. Note that the identity is expressed uniquely in this form: if we write $e=hk$, it follows that $h=k^{-1}$. Then since $K$ is a group, it must be the case that $h\in K$. However we know that $H\cap K=\{e\}$ from property \textbf{(c)}, so $h=e$ and therefore $k=e$ as well. Now to see that the rest of the elements in $g$ are expressed uniquely as $hk$, suppose $g$ is expressed as both $h_1k_1=h_2k_2$ for $h_1,h_2\in H$ and $k_1,k_2\in K$. Then $$e=gg^{-1}=(h_1k_1)(h_2k_2)^{-1}=h_1k_1k_2^{-1}h_2^{-1},$$ and from property $\textbf{(b)}$ we have that $$h_1k_1k_2^{-1}h_2^{-1}=h_1k_1h_2^{-1}k_2^{-1}=h_1h_2^{-1}k_1k_2^{-1}.$$ Now since $H$ and $K$ are closed, $h_1h_2^{-1}\in H$ and $k_1k_2^{-1}\in K$, so again the identity is being expressed as the product of an element of $H$ and of $K$. As shown above, it must be the case that $h_1h_2^{-1}=e$ and $k_1k_2^{-1}=e$. Cancellation on the right shows that $h_1=h_2$ and $k_1=k_2$, so the expression $g=hk$ is unique for all $g\in G$.

Now we will show that $G$ is isomorphic to $H\times K$. Since each $g\in G$ can be expressed uniquely as $g=hk$, let us define the function $\varphi:G\to H\times K$ by $\varphi(g)=\varphi(hk)=(h,k).$ We know this function is bijective from the assertions above. If $g_1=h_1k_1$ and $g_2=h_2k_2$ such that $\varphi(g_1)=\varphi(g_2)$, then $ (h_1,k_1)=(h_2, k_2)$ which means $h_1=h_2$ and $k_1=k_2$, and we know those expressions are unique so $g_1=g_2$. Also, since $G$ is closed we know that $\varphi$ is onto $H\times K$; for any $(h,k)\in H\times K$, we know $h,k\in G$ so the product $hk\in G$ and $\varphi(hk)=(h,k)$. To see the homomorphism property, we let $g_1,g_2\in G$ where $g_1=h_1k_1$ and $g_2=h_2k_2$. Then $$\varphi(g_1g_2)=\varphi(h_1k_1h_2k_2),$$ and again from property \textbf{(b)} (and associativity) we can rearrange the product $k_1h_2=h_2k_1$ such that $$\varphi(h_1k_1h_2k_2)=\varphi(h_1h_2k_1k_2)=(h_1h_2,k_1k_2)=(h_1,k_1)(h_2,k_2)=\varphi(g_1)\varphi(g_2).$$
Therefore if $G$ has subgroups $H$ and $K$ satisfying properties \textbf{(a)},\textbf{(b)}, and \textbf{(c)} then $G\cong H\times K$.
\end{proof}



\end{document}
