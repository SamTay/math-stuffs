
\documentclass{article}
\newenvironment{problem2}[1]{\noindent {\bf (#1}}
{\medskip}
\newenvironment{definition}[1][Definition]{\begin{trivlist}
\item[\hskip \labelsep {\bfseries #1}]}{\end{trivlist}}
\newenvironment{problem1}[1]{\noindent {\bf Problem #1:}}
{\medskip}
\usepackage{graphicx}
\usepackage{amsmath,amssymb,amsthm,amsfonts,graphicx,url,colordvi}
\usepackage{amssymb}
\usepackage{mathtools}
\makeatletter
\def\imod#1{\allowbreak\mkern8mu{\operator@font mod}\,\,#1}
\def\pimod#1{\allowbreak\mkern10mu{(\operator@font mod}\,\,#1)}
\makeatother

\usepackage{hyperref}
\usepackage{amssymb}
%\usepackage{fullpage}
\hypersetup{pdfborder={0 0 0}}
\DeclareMathOperator{\lcm}{lcm}
\DeclareMathOperator{\ran}{Ran}


\title{Math 327: Final Take-home Solutions}
\author{by Sam Tay}
\date{\today}

\begin{document}
%\maketitle
\begin{flushright}Sam Tay\\ Professor Aydin \\ Math 435 \\ Take-Home Midterm 1\\ 02/17/12
\end{flushright}
\vspace{.7cm}

\begin{problem1}{1} Suppose $R$ and $S$ are rings with unities $1_R$ and $1_S$ respectively, and $\theta:R\to S$ is a ring isomorphism.
\end{problem1}

\begin{problem2}{a)} Show that $\theta(1_R)=1_S$.
\begin{proof} Since $\theta$ is an isomorphism, we know that for any $b\in S$ there exists $\theta^{-1}(b)\in R$ where $\theta^{-1}(b)=1_R\cdot \theta^{-1}(b) = \theta^{-1}(b) \cdot 1_R$. Evaluating each side of this equality under $\theta$, it follows from the homomorphism property that $$b=\theta(1_R)\cdot b = b\cdot\theta(1_R).$$ This holds for all $b\in S$ and since the multiplicative identity satisfies this property uniquely, we conclude that $\theta(1_R)=1_S$.
\end{proof}
\end{problem2}
%*******************************************************************************
\begin{problem2}{b)} Show that $r$ is a unit in $R$ if and only if $\theta(r)$ is a unit in $S$.
\begin{proof} By definition, $r\in R$ is a unit means that there exists $r^{-1}\in R$ so that $r\cdot r^{-1}=r^{-1}\cdot r = 1_R$. Since $\theta$ is an isomorphism, we must have $$\theta(r)\cdot\theta(r^{-1})=\theta(r^{-1})\cdot\theta(r)=\theta(1_R)=1_S,$$ which means $\theta(r)$ has a multiplicative inverse $\theta(r)^{-1}=\theta(r^{-1})$ and is therefore a unit in $S$.

Conversely, suppose $\theta(r)$ is a unit in $S$ with multiplicative inverse $\theta(r)^{-1}$ so that $$\theta(r)\cdot\theta(r)^{-1}=\theta(r)^{-1}\cdot\theta(r)=1_S=\theta(1_R).$$ Again, we evaluate under $\theta^{-1}$ to find that for some $a\in R$, $$r\cdot a =a\cdot r=1_R$$ which allows us to conclude that $r$ has multiplicative inverse $a$ and is therefore a unit in $R$.
\end{proof}
\end{problem2}
%*******************************************************************************

\begin{problem2}{c)} For a ring $R$ with unity, let $U(R)$ denote the set of all units in $R$. Show that $U(R\times S)=U(R)\times U(S)$.
\begin{proof} ($\subseteq$) Suppose $(a,b)\in U(R\times S)$. Then $(a,b)$ has a multiplicative inverse $(c,d)\in R\times S$ so that $(a,b)(c,d)=(ac,bd)=(1_R,1_S)$. Then $ac=1_R$ and $bd=1_S$, so $a$ and $b$ are units in $R$ and $S$ respectively and we have $(a,b)\in U(R)\times U(S)$.\\
$(\supseteq)$ On the other hand, suppose $(a,b)\in U(R)\times U(S).$ Then $a$ and $b$ are units in $R$ and $S$ respectively, and have  multiplicative inverses $a^{-1}\in R$ and $b^{-1}\in S$. Then we have $(a^{-1},b^{-1})\in R\times S$ and obtain $$(a,b)(a^{-1},b^{-1})=(aa^{-1},bb^{-1})=(1_R,1_S).$$ Thus $(a,b)$ is a unit in $R\times S$ and $(a,b)\in U(R\times S)$, which completes the proof.
\end{proof}
\end{problem2}


\begin{problem2}{d)} Using part \textbf{(c)} and the Chinese Remainder Theorem (Example 18.15), what can you say about $\phi(mn)$ when $m$ and $n$ are relatively prime positive integers and $\phi$ is the Euler phi-function.
\begin{proof} Recall that $\phi:\mathbb{Z}^+\to\mathbb{Z}^+$ is defined such that $\phi(n)$ is the number of positive integers less than $n$ and relatively prime to $n$. By Theorem 20.12, these are precisely the integers $a\in \mathbb{Z}_n$ such that the equation $ax=1$ has a unique solution in $\mathbb{Z}_n$. It is now apparent that $\phi(n)$ is the number of units of $\mathbb{Z}_n$, so we write $$\phi(n)=\big|U(\mathbb{Z}_n)\big|.$$ For relatively prime integers $n,m$, Example 18.15 states that $\mathbb{Z}_{nm}\cong \mathbb{Z}_n\times\mathbb{Z}_m$. Clearly the number of units in isomorphic structures is the same, so that $$|U(\mathbb{Z}_{nm})|=|U(\mathbb{Z}_n\times\mathbb{Z}_m)|=|U(\mathbb{Z}_n)\times U(\mathbb{Z}_m)|=|U(\mathbb{Z}_n)||U(\mathbb{Z}_m)|,$$ where the penultimate equality follows from part \textbf{(c)}. From the reasoning above we can conclude that $\phi(mn)=\phi(m)\phi(n)$ for relatively prime integers $n,m$.
\end{proof}
\end{problem2}


\begin{problem2}{e)} Let $p$ be a prime and $m$ be a positive integer. Compute $\phi(p^m).$\\\\
To compute $\phi(p^m)$ we first note that since $p$ is a prime, the only integers $1\le n\le p^m$ that are not relatively prime to $p^m$ are multiples of $p$. How many multiples are there? We can count $$p,2p,3p,\ldots,p^{m-1}p=p^m,$$ from which it is clear that there are $p^{m-1}$ of them. This is the number of integers $1\le n \le p^m$ that are \emph{not} relatively prime to $n$, and we conclude that $\phi(p^m)=p^m-p^{m-1}.$
\end{problem2}

\begin{problem2}{f)} Now find a formula $\phi(n)$ for any integer $n>1$. Calculate $\phi(2700)$ using your formula.\\\\
For any integer $n>1$ with unique prime factorization $n=p_1^{k_1}p_2^{k_2}\cdots p_r^{k_r}$, we have \begin{align*}\phi(n) &= \phi(p_1^{k_1}p_2^{k_2}\cdots p_r^{k_r})\\
&=\phi(p_1^{k_1})\phi(p_2^{k_2})\cdots \phi(p_r^{k_r}) \quad\quad\quad\quad\quad\quad\quad\quad\quad\quad\quad \text{ by part \textbf{(d)} } \\
&=(p_1^{k_1}-p_1^{k_1-1})(p_2^{k_2}-p_2^{k_2-1})\cdots(p_r^{k_r}-p_r^{k_r-1}) \quad\quad \text{ by part \textbf{(e)} } \\
&=p_1^{k_1}p_2^{k_2}\cdots p_r^{k_r}\Big(1-\frac{1}{p_1}\Big)\Big(1-\frac{1}{p_2}\Big)\cdots\Big(1-\frac{1}{p_r}\Big)\\
&=n\;\Big(1-\frac{1}{p_1}\Big)\Big(1-\frac{1}{p_2}\Big)\cdots\Big(1-\frac{1}{p_r}\Big).
\end{align*}
To calculate $\phi(2700)$, we note that $2700=3^3\cdot2^2\cdot5^2,$ so $$\phi(2700)=2700\Big(1-\frac{1}{3}\Big)\Big(1-\frac{1}{2}\Big)\Big(1-\frac{1}{5}\Big)=720.$$
\end{problem2}
\vspace{2cm}

\begin{problem1}{2}\begin{definition}An element $a$ of a ring $R$ is nilpotent if $a^n=0$ for some positive integer $n$.\end{definition}
\end{problem1}

\begin{problem2}{a)} Let $a$ be a nilpotent element. Show that $a$ is either 0 or a zero divisor.
\begin{proof} Obviously $0$ is nilpotent, as $0^n=0$ for all positive integers $n$. For a nonzero nilpotent element $a\in R$, let $n$ be the least positive integer such that $a^n=0$. Since $a$ is nonzero, $n>1$. Then we have $a\cdot a^{n-1}=a^n =0,$ and since $n$ is the least such positive integer, the factors $a$ and $a^{n-1}$ are both nonzero. We conclude that $a$ is a zero divisor.
\end{proof}
\end{problem2}


\begin{problem2}{b)} In a ring $R$ with unity 1, prove that if $a$ is nilpotent, then $1-a$ is invertible.
\begin{proof} Let $a$ be nilpotent so that $a^n=0$ for some positive integer $n$. Then $$1=1-0=1-a^n=(1-a)(1+a+a^2+\cdots+a^{n-1}),$$ which is just the formula for the finite geometric sum. The equation above tells us that $1-a$ is invertible.
\end{proof}
\end{problem2}

\begin{problem2}{c)} In a commutative ring $R$, the product $xa$ of a nilpotent element $a$ by any element $x$ is nilpotent.
\begin{proof} Suppose that $a$ is nilpotent and let $n$ be a positive integer satisfying $a^n=0$. Then for any $x\in R$, $$(xa)^n=\overbrace{(xa)(xa)\cdots(xa)}^{n \text{ times }}=x^na^n$$ since $R$ is commutative. But $a^n=0$, so $(xa)^n=x^na^n=x^n\cdot0=0,$ and hence the product $xa$ is also nilpotent.
\end{proof}
\end{problem2}


\begin{problem2}{d)} In a commutative ring $R$, the sum of two nilpotent elements is nilpotent.
\begin{proof} Suppose that $a,b\in R$ are nilpotent elements with positive integers $n,m$ satisfying $a^n=0$ and $b^m=0$. Since $R$ is commutative we can use the binomial theorem to expand $(a+b)^{nm}$ as follows: $$(a+b)^{nm}=\sum_{k=0}^{nm}{nm \choose k} a^{k} b^{nm-k}.$$ First note that if $n=1$ then of course $a=0$ and $a+b=b$ is nilpotent. So suppose that $n>1$ and let us consider the terms of the sum above. When $k\ge n$, we see that the terms ${nm\choose k}a^kb^{nm-k}$ contain a factor of $a^n$ and therefore evaluate to zero. For the terms when $k<n$, we must have $-k>-n$, from which it follows that $nm-k>nm-n>m-1$, where the last inequality follows from our assumption that $n>1$. Thus $nm-k\ge m$ and the terms ${nm\choose k}a^kb^{nm-k}$ contain a factor of $b^m$ and also evaluate to zero. We have shown that all terms in the binomial expansion above must be zero, so $(a+b)^{nm}=0$ and we conclude $a+b$ is nilpotent.
\end{proof}
\end{problem2}

\begin{definition}An element $a$ of a ring is unipotent if $1-a$ is nilpotent.\end{definition}

\begin{problem2}{e)} In a commutative ring $R$ the product of two unipotent elements is unipotent.
\begin{proof} Let $a,b\in R$ be unipotent elements. To show that the product $ab$ is unipotent, we must show that $1-ab$ is nilpotent. We see that $$(1-ab)=(1-a)+a(1-b),$$ where $(1-a)$ and $(1-b)$ are both nilpotent. By part \textbf{(c)}, the product $a(1-b)$ is also nilpotent, and then by part \textbf{(d)} the sum $(1-a)+a(1-b)$ is nilpotent. Therefore $(1-ab)$ is nilpotent and $ab$ is unipotent.
\end{proof}
\end{problem2}



\begin{problem2}{f)} In a ring $R$ with unity 1, every unipotent element is invertible. 
\begin{proof} Let $a\in R$ be unipotent. Then $(1-a)$ is nilpotent, so by part \textbf{(b)} we know $1-(1-a)$ is invertible, where $1-(1-a)=1-1-(-a)=a$, which completes the proof.
\end{proof}
\end{problem2}



\begin{problem1}{3} Let $R$ be the set of all functions from $\mathbb{R}$ to $\mathbb{R}$. We define addition and multiplication on these functions in the usual way.\end{problem1}

\begin{problem2}{a)} Describe the additive and multiplicative identities in the ring $R$.\\

The additive identity is the constant zero function $f(x)=0$. The multiplicative identity is the constant function $f(x)=1$.\end{problem2}\\

\begin{problem2}{b)} What are the units of $R$?\\

The units of $R$ are the functions satisfying $f(x)\ne0$ for all $x\in\mathbb{R}$. We see that given this constraint, we can construct a well defined function $g:\mathbb{R}\to\mathbb{R}$ defined by $g(x)=\frac{1}{f(x)},$ so that $f(x)g(x)=1$ for all $x\in \mathbb{R}$. On the other hand, because $\mathbb{R}$ has no zero divisors, if a function $h$ has a root $h(x_0)=0$ then there is no $r\in\mathbb{R}$ such that $r\cdot h(x_0)=1$, and therefore no function $g(x)$ such that $h(x_0)g(x_0)=1$. Therefore the units are precisely the functions satisfying $f(x)\ne0$ for all $x\in\mathbb{R}$.
\end{problem2}\\


\begin{problem2}{c)} Determine all zero divisors in $R$.\\

Let $f,g\in R$. Since $\mathbb{R}$ has no zero divisors, the product $(fg)(x)=f(x)g(x)$ is the zero function if and only if for each $x_0\in\mathbb{R}$, either $f(x_0)=0$ or $g(x_0)=0$. Thus $fg=0$ if $(\ker f )\cup( \ker g) = \mathbb{R}$. However we are interested in the nonzero functions, so consider when $fg=0$ for nonzero $g$. Then there exists $x_0\in \mathbb{R}$ where $g(x_0)\ne0$, so we must have $f(x_0)=0.$ On the other hand, if $f(y_0)=0$ for some $y_0\in\mathbb{R}$, we can construct a nonzero function $$g(x)=\begin{cases} 0 & \text{ if $x\ne y_0$ }\\ 1 & \text{ if $x=y_0$ }\end{cases}$$ so that $fg=0$. We conclude that the zero divisors are precisely the nonzero functions $f$ with some root $x_0$ where $f(x_0)=0$. In regards to the set notation above, we could equivalently say that $f$ is a zero divisor if and only if $\emptyset\subset \ker f \subset \mathbb{R}$.
\end{problem2}\\

\begin{problem2}{d)} Determine all nilpotent elements in $R$.\\

Let $f\in R$ be a nilpotent  element. Then there exists a positive integer $n$ such that $f^n$ is the zero function, which means $(f(x_0))^n=0$ for all $x_0\in\mathbb{R}$. But $f(x_0)$ is just a real number, of which there are no zero divisors. Then it must be the case that $f(x_0)=0$ for all $x_0\in\mathbb{R}$, so the only nilpotent function in $R$ is the zero function.
\end{problem2}\\


\begin{problem2}{e)} Is the following statement true for $R$? \begin{center} ``Every nonzero element is either a zero divisor or a unit."\end{center}

This is true! For any nonzero function $f\in R$, either there exists a root $x_0$ where $f(x_0)=0$ or there does not. In the former case $f$ is a zero divisor by part \textbf{(c)}, and in the latter case $f$ is a unit by part \textbf{(b)}. \end{problem2}\\

\begin{problem2}{f)} Is the statement in part \textbf{(e)} true for a general ring with unity?\\

Of course not: consider the ring of integers $\mathbb{Z}$ with unity 1, which is an integral domain and thus has no divisors of zero. We know that the only units are 1,-1 so there are many elements that are neither units nor divisors of zero. For a concrete counterexample, we'll choose the element $64\in\mathbb{Z}$ which is nonzero, not a unit, and not a zero divisor.
\end{problem2}\\


\begin{problem1}{4} Let $R$ be a commutative ring and let $p(x)=a_nx^n+a_{n-1}x^{n-1}+\cdots+a_0\in R[x]$. Then $p(x)$ is a zero divisor in $R[x]$ if and only if there is a nonzero $b\in R$ such that $b\cdot p(x)=0$.
\begin{proof} Let $p(x)$ be defined as above and assume further that $a_n\ne0$, so that $p(x)$ has degree $n$. The backward direction is trivial: if $b\cdot p(x)=0$ for nonzero elements $b,p(x)\in R[x]$ then by definition $b$ and $p(x)$ are zero divisors. In the forward direction, we assume that $p(x)$ is a zero divisor. Then there exists at least one nonzero polynomial $g(x)\in R[x]$ so that $p(x)g(x)=0$. Let us pick $g(x)$ specifically to have minimal degree among such functions.\footnote{Specifically, we are considering the set $T=\{g(x)\in R[x] : p(x)g(x)=0 \text{ and } g(x)\text{ is nonzero}\}$ and choosing $g\in T$ such that $\deg(g)\le\deg(h)$ for all $h\in T$.} Without loss of generality, we'll suppose that $g(x)=b_mx^m+b_{m-1}x^{m-1}+\cdots+b_0,$ where $b_m\ne 0$ and $\deg(g)=m$. We wish to show that $a_{n-i}g(x)=0$ for each $0\le i\le n$, from which the claim will follow easily. To this end, we will proceed by strong induction on $i$. Computing the product $p(x)g(x)$, we find that the leading coefficient is $a_nb_m,$ which must be zero since $p(x)g(x)$ is the zero polynomial. Now consider the polynomial \begin{align*}a_ng(x)&=a_nb_mx^m+a_nb_{m-1}x^{m-1}+\cdots+a_nb_0\\ &=0x^m+a_nb_{m-1}x^{m-1}+\cdots+a_nb_0\\ &=a_nb_{m-1}x^{m-1}+\cdots+a_nb_0.\end{align*} Then by associativity, $$(a_ng(x))p(x)=a_n(g(x)p(x))=a_n\cdot 0=0.$$ However this polynomial $a_ng(x)$ has degree less than $g(x)$, which we defined to have minimal degree among the nonzero functions satisfying this equation. The only possibility is that $a_ng(x)$ is the zero polynomial, and our base case is now proven. Next suppose for induction that for some $0<k\le n$,\; $a_{n-i}\;g(x)=0$ for all $0\le i < k$. Again, we consider the product \begin{align*} 0=p(x)g(x)&=(a_nx^n+a_{n-1}x^{n-1}+\cdots+a_0)\,g(x) \\ &= a_nx^n g(x)+a_{n-1}x^{n-1}g(x)+\cdots+a_0g(x) \\
&=x^n(a_ng(x))+x^{n-1}(a_{n-1}g(x))+\cdots+a_0g(x)\\ &= a_{n-k}x^{n-k}g(x)+a_{n-k-1}x^{n-k-1}g(x)+\cdots+a_0g(x),\end{align*} where the last equality follows from our inductive hypothesis. Let us consider this last expression a bit more carefully. We see that the terms in this sum have degrees (individually of course) as below: 
$$0=p(x)g(x)=\overbrace{ a_{n-k}x^{n-k}g(x)}^{\deg \le m+n-k}+\overbrace{a_{n-k-1}x^{n-k-1}g(x)}^{\deg\le m+n-k-1}+\cdots+\overbrace{a_0g(x)}^{\deg\le m}.$$ In particular, we see that the only occurrence of $x^{n+m-k}$ is within the first term of highest degree; equally clear is that within this term, we find the leading coefficient $a_{n-k}b_m$, which as before must be equal to zero. Just as before, we find that $a_{n-k}g(x)$ has degree less than that of $g$: \begin{align*}a_{n-k}g(x)&=a_{n-k}b_mx^m+a_{n-k}b_{m-1}x^{m-1}+\cdots+a_{n-k}b_0\\&=0x^m+a_{n-k}b_{m-1}x^{m-1}+\cdots+a_{n-k}b_0\\ &=a_{n-k}b_{m-1}x^{m-1}+\cdots+a_{n-k}b_0.\end{align*} However $(a_{n-k}g(x))p(x)=0$ where $a_{n-k}g(x)$ is of lesser degree than $g$, so once again we conclude $a_{n-k}g(x)=0$. Therefore by induction on $i$ we have shown that $a_{n-i}g(x)=0$ for all $i=0,1,\ldots,n$. Explicitly, \begin{align*} &a_{n-i}(b_mx^m+b_{m-1}x^{m-1}+\cdots+b_0)=0\\
&\phantom{111}\Longrightarrow a_{n-i}b_m=0\quad\text{ for all $i=0,1,\ldots,n$.}
\end{align*}
Finally, recalling that $b_m$ was the \emph{nonzero} leading coefficient of $g$,\begin{align*}b_mp(x)&=b_m(a_nx^n+a_{n-1}x^{n-1}+\cdots+a_0)\\
&=b_ma_nx^n+b_ma_{n-1}x^{n-1}+\cdots+b_ma_0\\
&=0,\end{align*} and the proof is complete.\end{proof}
\end{problem1}

\begin{problem1}{5} Let $F$ be a field and let $a\in F^*$. \end{problem1}\\

\begin{problem2}{a)} If $af(x)$ is irreducible over $F$, prove that $f(x)$ is irreducible over $F$.
\begin{proof} Suppose $f(x)$ is reducible over $F$. Then there exist $g(x),h(x)\in F[x]$ so that $f(x)=g(x)h(x)$ and the degrees of $g,h$ are both less than that of $f$. Then $af(x)=ag(x)h(x)=\big(ag(x)\big)h(x)$, and since $a$ has degree 0, $$\deg(af)=\deg(f)\quad\quad \text{ and } \quad\quad \deg(ag)=\deg(g).$$ So again the degrees of $ag, h$ are less than that of $af$, and we conclude that $af$ is reducible over $F$.
\end{proof}\end{problem2}

\begin{problem2}{b)} If $f(ax)$ is irreducible over $F$, prove that $f(x)$ is irreducible over $F$.
\begin{proof} Suppose $f(x)$ is reducible over $F$ such that $f(x)=g(x)h(x)$ and $$\deg f> \deg g, \deg h. $$ Note that for any polynomial $p(x)=\sum_{i=0}^nc_ix^i\in F[x]$, $$p(ax)=\sum_{i=0}^nc_i(ax)^i=\sum_{i=0}^nc_ia^nx^n$$ and since there are no zero divisors, $\deg( p(ax))=\deg (p(x))$. Then we have $f(ax)=g(ax)h(ax)$ where $$\deg( f(ax)) > \deg (g(ax)), \deg( h(ax)),$$ so $f(ax)$ is reducible over $F$.
\end{proof}
\end{problem2}

\begin{problem2}{c)} If $f(x+a)$ is irreducible over $F$, prove that $f(x)$ is irreducible over $F$.
\begin{proof} Again, let $f(x)$ be reducible over $F$ such that $f=gh$ and the degrees of $g,h$ are less than that of $f$. Note that for any polynomial $p(x)=\sum_{i=0}^nc_ix^i\in F[x],$ $$p(x+a)=\sum_{i=0}^nc_i(x+a)^i=\sum_{i=0}^nc_i\Bigg[x^i + {i\choose1}x^{i-1}a+{i\choose2}x^{i-2}a^2+\cdots+a^i\Bigg].$$ We see that the leading term of $p(x+a)$ is still $c_nx^n$, so $p(x)$ and $p(x+a)$ have equal degree. Clearly then, $f(x+a)=g(x+a)h(x+a)$ is a product of lesser degree polynomials, so $f(x+a)$ is reducible over $F$.
\end{proof}
\end{problem2}

\begin{problem2}{d)} Use part \textbf{(c)} to show that $f(x)=8x^3-6x+1$ is irreducible over $\mathbb{Q}.$
\begin{proof} If we can find a function $f(x+a)$ that we know is irreducible over $\mathbb{Q}$, we can conclude from part \textbf{(c)} that $f(x)$ is irreducible over $\mathbb{Q}$ as well. We find that $$f(x+1)=8(x+1)^3-6(x+1)+1=8x^3+24x^2+18x+3.$$ Now we see that $f(x+1)$ satisfies the Eisenstein Criterion for prime $p=3$. Therefore by Theorem 23.15 $f(x+1)$ is irreducible over  $\mathbb{Q}$, and by part \textbf{(c)} we conclude that $f(x)$ is also irreducible over $\mathbb{Q}$.
\end{proof}
\end{problem2}


\begin{problem1}{6} Let $f(x)\in\mathbb{R}[x]$. If $f(a)=0$ and $f'(a)=0$, then $(x-a)^2\big\vert f(x).$
\begin{proof} Since $\mathbb{R}$ is a field, the Factor Theorem states that $f(a)=0$ implies $f(x)=g(x)(x-a)$ for some $g(x)\in\mathbb{R}[x]$. Therefore to show that $(x-a)^2\big\vert f(x)$, it suffices to show $(x-a)\big\vert g(x)$. Taking the derivative, we know from calculus that $$f'(x)=g'(x)(x-a)+g(x)(1).$$ Recalling that $f'(a)=0$, we find \begin{align*} f'(a)&=g'(a)(a-a)+g(a)\\
\Longrightarrow\phantom{111} 0&=0+g(a).\end{align*} So $a$ is a zero of $g(x)$ and again by the Factor Theorem $g(x)=h(x)(x-a)$ for some $h(x)\in \mathbb{R}[x]$. As mentioned above, this shows that $(x-a)^2\big\vert f(x)$.
\end{proof}

\end{problem1}


\end{document}











