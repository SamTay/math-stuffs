
\documentclass{article}
\newenvironment{problem2}[1]{\noindent {\bf (#1}}
{\medskip}

\newenvironment{problem1}[1]{\noindent {\bf Problem #1:}}
{\medskip}
\usepackage{graphicx}
\usepackage{amsmath,amssymb,amsthm,amsfonts,graphicx,url,colordvi}
\usepackage{amssymb}
\usepackage{mathtools}
\makeatletter
\def\imod#1{\allowbreak\mkern10mu({\operator@font mod}\,\,#1)}
\makeatother

\usepackage{hyperref}
\usepackage{amssymb}
%\usepackage{fullpage}
\hypersetup{pdfborder={0 0 0}}
\DeclareMathOperator{\lcm}{lcm}
\DeclareMathOperator{\ran}{Ran}


\title{Math 327: Final Take-home Solutions}
\author{by Sam Tay}
\date{\today}

\begin{document}
%\maketitle
\begin{flushright}Sam Tay\\ Professor Milnikiel \\ Math 335 \\ Section 13: 4, 6, 19, 22, 37, 39, 47, 52, 53\\ 12/23/11
\end{flushright}

\begin{problem1}{4} Let $\phi:\mathbb{Z}_6\to\mathbb{Z}_2$ be defined by $\phi(x) = x\imod{2}.$ To see that $\phi$ is a homomorphism, let $\ a,b\in\mathbb{Z}_6$ and compute $$\phi(a+_6b)= (a+ b \mod{6})\imod{2}$$ \begin{align*}\phi(a)+_2\phi(b)&=(a\mod{2})+(b\mod{2}) \imod{2} \\
&=a+b\imod{2}. \end{align*} Therefore, to show that $\phi(a+_6b)=\phi(a)+_2\phi(b)$, we must show $$a+b \imod{6} \equiv_2 a+b.$$ Using the Divisor Theorem to express $a+b=6m+r,$ where $0\le r<6$, we see that $a+b \imod{6} = r$. Now $$a+b \equiv_2 6m +r \equiv_2 2(3m) + r \equiv_2 r,$$ which completes the proof. 
\end{problem1}

\begin{problem1}{6} Let $\phi:\mathbb{R}\to\mathbb{R}^*,$ where $\mathbb{R}$ is additive and $\mathbb{R}^*$ is multiplicative, be given by $\phi(x)=2^x$. To see that $\phi$ is a homomorphism, we let $x,y\in\mathbb{R}$ and compute $$\phi(x+y)=2^{x+y}=2^x2^y=\phi(x)\phi(y).$$
\end{problem1}


\begin{problem1}{19} We have a group homomorphism $\phi:\mathbb{Z}\to S_8$ such that $\phi(1)=(1, 4, 2, 6)(2, 5, 7).$ Note that $(1, 4, 2, 6)(2, 5, 7)=(1, 4, 2, 5, 7, 6)$, from which it is clear that $|\phi(1)|=6$. This means that $i=\phi(1)^6=\phi(6)$, where $i$ is the identity permutation and the second equality follows from the homomorphism property. However, we know that for any $n\in\mathbb{Z}$, we have $i^n=i$. Thus $$i=\phi(6)^n=\phi(6n)$$ for all $n\in\mathbb{Z}$, and since $6$ is the least such positive integer, $\ker\phi=6\mathbb{Z}$. Now to compute $\phi(20)$, we use the previous assertions to find $$\phi(20)=\phi(6\cdot3 + 2)=\phi(6)^3\phi(2)=i^3\phi(2)=\phi(2).$$ It is easy to see that $$\phi(20)=\phi(2)=(1, 4, 2, 5, 7, 6)(1, 4, 2, 5, 7, 6)=(1, 2, 7)(4, 5, 6).$$
\end{problem1}


\begin{problem1}{22} We have a group homomorphism $\phi:\mathbb{Z}\times\mathbb{Z}\to \mathbb{Z}$ such that $\phi(1,0)=3$ and $\phi(0,1)=-5.$ Then $(z_1,z_2)\in\ker\phi$ if and only if $\phi(z_1,z_2)=0$, where $$\phi(z_1,z_2)=\phi\big((z_1,0)+(0,z_2)\big)=\phi(z_1,0)+\phi(0,z_2),$$ and with $z_1+z_2$ applications of the homomorphism property, $$\phi(z_1,z_2)=z_1\phi(1,0)+z_2\phi(0,1)=3z_1-5z_2.$$ Therefore $$\ker\phi=\{(z_1,z_2)\in\mathbb{Z}\times\mathbb{Z} : 3z_1=5z_2\}.$$ From above, we see that $\phi(-3,2)=3(-3)-5(2)=-19.$
\end{problem1}

\begin{problem1}{37} Let $\phi:\mathbb{Z}_3\to S_3$ be defined by $$\phi(n)=\begin{cases} i & \text{ if $n=0$} \\ (1, 2, 3) & \text{ if $n=1$ } \\ (1, 3, 2) & \text{ if $n=2$ }. \end{cases} $$ The homomorphism property is obviously satisfied for operations involving the identity 0, so let us check the cases with $1,2\in\mathbb{Z}_3$. We have $$\phi(1)\phi(2)=(1,2,3)(1,3,2)=i=\phi(0)=\phi(1+2)$$ $$\phi(2)\phi(1)=(1,3,2)(1,2,3)=i=\phi(0)=\phi(2+1)$$ $$\phi(1)\phi(1)=(1,2,3)(1,2,3)=(1,3,2)=\phi(2)=\phi(1+1)$$ $$\phi(2)\phi(2)=(1,3,2)(1,3,2)=(1,2,3)=\phi(1)=\phi(2+2),$$ and indeed $\phi$ is a nontrivial homomorphism.
\end{problem1}

\begin{problem1}{39} Define $\phi:\mathbb{Z}\times\mathbb{Z}\to2\mathbb{Z}\;$ by $\phi(z_1,z_2)=2(z_1+z_2).$ For $(x_1,y_1),( x_2,y_2)\in\mathbb{Z}\times\mathbb{Z}$, we have  $$\phi\big((x_1,y_1)+(x_2,y_2)\big)=\phi(x_1+x_2,y_1+y_2)=2(x_1+x_2+y_1+y_2),$$ and $$\phi(x_1,y_1)+\phi(x_2,y_2)=2(x_1+y_1)+2(x_2+y_2)=2(x_1+x_2+y_1+y_2),$$ and thus $\phi$ is a nontrivial homomorphism.
\end{problem1}


\begin{problem1}{47} If $\phi:G\to G'$ is a group homomorphism and $|G|$ is prime, then $\phi$ is trivial or one-to-one.
\begin{proof} Since $|G|$ is finite, $|\phi[G]|$ is a finite divisor of $|G|$ by Exercise 44. Since $|G|$ is prime, this means $|\phi[G]|=1$ or $|\phi[G]|=|G|$. We know $e'\in\phi[G]$ so in the former case, $\phi$ is trivial. In the latter case, if $|G|$ distinct elements map to $|G|$ distinct elements, clearly $\phi$ is one-to-one (if not, then we would have $|\phi[G]|<|G|$).\end{proof}\end{problem1}

\begin{problem1}{52} Let $\phi: G\to G'$ be a homomorphism with kernel $H$ and let $a\in G$. Then $\{x\in G: \phi(x)=\phi(a)\}=Ha$.
\begin{proof}$(\subseteq)$ Suppose $x\in G$ such that $\phi(x)=\phi(a).$ Then $\phi(x)\phi(a)^{-1}=e'$ and by Theorem 13.12, $\phi(a)^{-1}=\phi(a^{-1}).$ It follows then by the homomorphism property that $e'=\phi(x)\phi(a^{-1})=\phi(xa^{-1}).$ Therefore $xa^{-1}$ is in the kernel such that $xa^{-1}=h$ for some $h\in H$. Since $G$ is a group, $x=ha$ and we conclude $x\in Ha$.

$(\supseteq)$ Next suppose $x\in Ha$, such that $x=ha$ for some $h\in H$. Then $$\phi(x)=\phi(ha)=\phi(h)\phi(a)=e'\phi(a)=\phi(a),$$ so $x\in \{x\in G : \phi(x)=\phi(a)\}$, and this completes the proof.
\end{proof}
\end{problem1}

\begin{problem1}{53} Let $G$ be a group with $h,k\in G$ and define $\phi:\mathbb{Z}\times\mathbb{Z}\to G$ by $\phi(m,n)=h^mk^n$. Then $\phi$ is a homomorphism if and only if $hk=kh$.
\begin{proof} $(\Leftarrow)$ Suppose $hk=kh$. Then for all $n,m\in\mathbb{Z}$, $h^mk^n=k^nh^m$. To see this, first note that $$h^mk=h^{m-1}hk=h^{m-1}kh=h^{m-2}hkh=h^{m-2}kh^2=\cdots=kh^m.$$ Using this result, we find similarly that $$h^mk^n=h^mkk^{n-1}=kh^mk^{n-1}=kh^mkk^{n-2}=k^2h^mk^{n-2}=\cdots=k^nh^m.$$ Now let $m_1,n_1,m_2,n_2\in\mathbb{Z}$ and we have \begin{align*}\phi(m_1,n_1)\phi(m_2,n_2)&=h^{m_1}k^{n_1}h^{m_2}k^{n_2}=h^{m_1}h^{m_2}k^{n_1}k^{n_2}\\ &=h^{m_1+m_2}k^{n_1+n_2}=\phi(m_1+m_2,n_1+n_2)\\ &=\phi\big((m_1,n_1)+(m_2,n_2)\big). \end{align*} Therefore $\phi$ is a homomorphism.

$(\Rightarrow)$ Next suppose $\phi$ as defined above is a homomorphism. Then for any $m_1,n_1,m_2,n_2\in\mathbb{Z}$ we must have $\phi(m_1,n_1)\phi(m_2,n_2)=\phi\big((m_1,n_2)+(m_2,n_2)\big)$, where $$\phi(m_1,n_1)\phi(m_2,n_2)=h^{m_1}k^{n_1}h^{m_2}k^{n_2}$$ $$\phi\big((m_1,n_1)+(m_2,n_2)\big)=\phi(m_1+m_2,n_1+n_2)=h^{m_1+m_2}k^{n_1+n_2}.$$ Then $$h^{m_1}k^{n_1}h^{m_2}k^{n_2}=h^{m_1}h^{m_2}k^{n_1}k^{n_2},$$ and since $G$ is a group, left and right cancellation gives $$k^{n_1}h^{m_2}=h^{m_2}k^{n_1}.$$ Since this must hold for all integers, it must hold for $n_1=m_2=1$, so that $hk=kh$.
\end{proof}
\end{problem1}
%%%%%%%%%%%%%%%%%%%%%%%%%%%%%%%%%%%%%%%%%%%%


\begin{center}\textbf{Questions from the Unstarred Problems}\\\end{center}



\begin{problem1}{24} Would the answer change if the permutations $\phi(1,0)$ and $\phi(0,1)$ were not independent cycles (and therefore did not commute)? As the problem is written, I found $$\ker\phi=\{(2n,4m) : n,m\in\mathbb{Z}\},$$ which follows because  $|\phi(1,0)|=2$ and $|\phi(0,1)|=4$. My thinking is that it wouldn't change the image $\phi^{-1}[\{i\}]=\ker\phi$, but that $\phi$ would no longer be a homomorphism... is that right?
\end{problem1}

\begin{problem1}{26} For a function $\phi:\mathbb{Z}\to\mathbb{Z}$, suppose $\phi(1)=a$. Then $\phi(z)=az$ for all $z\in\mathbb{Z}$ and $\phi$ is a homomorphism. Since $a$ was arbitrary, there are countably infinitely many homomorphisms from  $\mathbb{Z} $ into $ \mathbb{Z}$.
\end{problem1}

\begin{problem1}{32 (a)} Is $A_n$ a normal subgroup of $S_n$? I would think not, but only because the elements of $A_n$ don't generally commute with the rest of $S_n$. Is there a better reason?
\end{problem1}

\begin{problem1}{32 (j)} Is it possible to have a nontrivial homomorphism of a finite group into an infinite group? I think you could if the elements were sent to some finite subgroup of the codomain. Perhaps a function $\phi:\mathbb{Z}_n\to U$, where $U$ is the infinite set of complex numbers of unit magnitude, but the $\ran\phi=U_n$.
\end{problem1}

\begin{problem1}{33} I don't think it's possible to have a nontrivial homomorphism, but I have trouble showing it. I think, in regards to Problem 4, we would need $$a+b\imod{12}\equiv_5 a+b,$$ and using the Divisor Theorem to write $a+b=12m+r$, we need $$a+b\equiv_5 12m+r\equiv_5 r,$$ but if $m|5$ then this can hold. Then it follows that for all $c\in\mathbb{Z}_{12}$, we must have $c=60n +r$ for some $n,r\in\mathbb{Z}$. However can't we choose representatives for $\mathbb{Z}_{12}$ to make this possible? I'm confused.
\end{problem1}

\begin{problem1}{34} If we define $\phi:\mathbb{Z}_{12}\to\mathbb{Z}_4$ by $\phi(a)=a\imod{4}$, do we have a homomorphism?
\end{problem1}

\begin{problem1}{35} Define $\phi:\mathbb{Z}_{2}\times\mathbb{Z}_4\to\mathbb{Z}_2\times\mathbb{Z}_5$ by $\phi(z_1,z_2)=(z_1,0)$, then $\phi$ is a homomorphism.
\end{problem1}

\begin{problem1}{36} There are no nontrivial homomorphisms from $\mathbb{Z}_3$ into $\mathbb{Z}$ because if $\phi(1)=z$, then $$3z=z+z+z=\phi(1)+\phi(1)+\phi(1)=\phi(0)=0,$$ such that $z=0$. It follows that $\ran\phi=0$.
\end{problem1}

\begin{problem1}{41} This took me forever to figure out. I feel that there must be a better way to think about these problems; don't really have any intuition for questions like these. I think, however, that we can define a nontrivial homomorphism $\phi:D_4\to S_3$ by defining the function as follows: $$\phi(x)=\begin{cases} i & \text{ if $x$ is $\rho_0,\rho_1,\rho_2,$ or $\rho_3$} \\ (1, 2) & \text{ if $x$ is $\mu_1,\mu_2, \delta_1,$ or $\delta_2$}.\end{cases} $$ Does that look good to you?\end{problem1}

\begin{problem1}{42} For $\phi:S_3\to S_4$, can we just send $\phi(\sigma)=\sigma$, such that the fourth element of $S_4$ is left alone?\end{problem1}

\begin{problem1}{51} Let $G$ be a group and $a\in G$. Then $\phi:\mathbb{Z}\to G$ defined by $\phi(n)=a^n$ is a homomorphism. This is easy enough to show, but in exploring the properties of the kernel, do I have these things correct: \begin{enumerate} \item{ $\ran\phi=\langle a \rangle$.} \item{If $\ker\phi=\{0\}$ then $\langle a \rangle$ is infinite.}\item{ If $z\in\ker\phi$ then $kz\in\ker\phi$ for all $k\in\mathbb{Z}$.}\end{enumerate}
\end{problem1}




\end{document}



























