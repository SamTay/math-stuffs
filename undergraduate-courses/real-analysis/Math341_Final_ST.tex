%%This is a standard LaTeX2e article document template. personal version 12/5/200%%

\documentclass[12pt,twoside]{article}

%%%%%%%%%%%%%%%%%%%%%%%%%%%%%%%packages%%%%%%%%%%%%%%%%%%%%%%%%%%%%%%%%%%%%%%%%%%%%%%%%%%%%%%%%%%
\usepackage{amsfonts}
\usepackage{latexsym}
\usepackage{amssymb}
\usepackage{amstext}
\usepackage{multicol}
\usepackage{amsmath}
\usepackage{amsthm}
%%%%%%%%%%%%%%%%%%%%%%%%%%%%%%%formatting%%%%%%%%%%%%%%%%%%%%%%%%%%%%%%%%%%%%%%%%%%%%%%%%%%%%%%%
\pagestyle{plain}
\setlength{\topmargin}{0in}        %%%  This sets all the spacing stuff to use the page more
\setlength{\oddsidemargin}{0in}    %%%  efficiently than the normal "article" setup would.
\setlength{\evensidemargin}{0in}   %%%  It's OK to play with these some.
\setlength{\textheight}{8.5in}     %%%
\setlength{\textwidth}{6.25in}     %%%
\setlength{\headsep}{0in}          %%%
\setlength{\headheight}{0in}       %%%
%\setlength{\footskip}{0in}        %%%

%%%%%%%%%%%%%%%%%%%%%%%%%%%%%%%new commands%%%%%%%%%%%%%%%%%%%%%%%%%%%%%%%%%%%%%%%%%%%%%%%%%%%%%%

\newcommand{\Int}[1]{{\cal I}nt( #1)}
\newcommand{\spifff}{\qquad\text{iff}\qquad}
\newcommand{\spand}{\qquad\text{ and }\qquad}
\newcommand{\imp}{\Rightarrow}
\newcommand{\ifff}{\Leftrightarrow}
%\newcommand{\ker}{\text{ker}}
\newcommand{\hull}{\text{hull}\,}
\newcommand{\Lan}{\mathcal{L}}
\makeatletter
\def\imod#1{\allowbreak\mkern10mu({\operator@font mod}\,\,#1)}
\makeatother

\newcommand{\forward}{\noindent ($\Longrightarrow$) \,\,}
\newcommand{\back}{\noindent ($\Longleftarrow$) \,\,}
\newcommand{\R}{\mathbb R} %REALS
\newcommand{\C}{\mathbb C} %COMPLEX
\newcommand{\N}{\mathbb N} %NATURAL NUMBERS
\newcommand{\Q}{\mathbb Q} %RATIONALS
\newcommand{\Z}{\mathbb Z} %INTEGERS


%\theoremstyle{definition}

\newenvironment{cexample}{\noindent {\em Counterexample:}}{\bigskip}
\newenvironment{example}{\noindent {\em Example:} }{\bigskip}
\newenvironment{solution}{\noindent {\em Solution:} \\}{\hfill
\rule{1mm}{3mm}\bigskip}
\DeclareMathOperator*{\diam}{diam}


\pagestyle{empty}
%%%%%%%%%%%%%%%%%%%%%%%%%%%%%%%%%%%%%%%%%%%%%%%%%%%%%%%%%%%%%%%%%%%%%%%%%%%%%%%%%%%%%%%%%%%%%%%


\begin{document}

\begin{center}
Sam Chong Tay\\
Math 341: Real Analysis\\
Take-home Final Exam\\
December 19, 2012\\
\end{center}

\begin{enumerate}
\item (15 pts) Let $A$ and $B$  be non-empty subsets of $\R$ with $A\subseteq B$.  Suppose that $\inf A$, $\inf B$, and $\sup B$ all exist.

\begin{itemize}
\item[(a)] Prove that $\sup A$ exists.

\begin{proof} We know $A$ is non-empty by hypothesis, so we only need to show that $A$ is bounded above. Considering any $a\in A$, we see that $a\in B$ as well because $A\subseteq B$. Thus $a \le \sup B$, and since $a$ was arbitrary we have shown that $A$ is bounded above by $\sup B$. Hence by Axiom III, $\sup A$ exists.
\end{proof}

\item[(b)] Prove that $$ \inf(B)\leq\inf(A)\leq\sup(A)\leq\sup(B). $$

\begin{proof} For the first inequality, it suffices to show that $\inf(B)$ is a lower bound for $A$. So, similar to part (a), we note that for any $a\in A$, $a$ is also in $B$ and therefore $\inf(B) \le a$. Hence $\inf(B)$ is a lower bound for $A$ and since $\inf(A)$ is the greatest lower bound, we have $$\inf(B) \le \inf(A).$$ The next inequality is quite trivial; since $A$ is non-empty there exists $a\in A$ and by definition of infimum and supremum we have immediately that $$\inf(A)\le a \le \sup(A).$$ Finally, note that we showed in part (a) that $\sup(B)$ is an upper bound for $A$. Therefore just as in the infimum case, since $\sup(A)$ is the least upper bound, we have $\sup(A)\le\sup(B)$. Altogether we have shown that $$ \inf(B)\leq\inf(A)\leq\sup(A)\leq\sup(B).$$\end{proof}

\item[(c)] Show by giving a counterexample that $$\inf(B)=\inf(A)\qquad\textrm{and}\qquad \sup(A)=\sup(B)$$ need not imply that $A=B$.\\

\begin{cexample} If we define $$A=\{0,1\} \spand B=[0,1]$$
we see that $$ \inf(A)=\inf(B)=0 \spand \sup(A)=\sup(B)=1,$$ but of course $A\ne B$.
\end{cexample}

\end{itemize}


\item (15 pts) Prove that every non-convergent bounded sequence of real numbers has at least two limit points.

\begin{proof} Let $(s_n)$ be a non-convergent bounded sequence in $\R$. Because $(s_n)$ is bounded we know from Theorem 3.4.9 that $(s_n)$ has at least one convergent subsequence, and hence at least one limit point $x$. Consider an arbitrary subsequence $(s_{n_k})$ of $(s_n)$. This subsequence must also be bounded and thus by the same theorem has a convergent subsequence $(s_{n_{k_i}})$. Well $(s_{n_{k_i}})$ is evidently also a subsequence of $(s_n)$ in its own right, so we see that if $x$ was the only limit point, this would imply $s_{n_{k_i}}\to x$. But we chose an arbitrary subsequence, so we have shown that if $x$ is the only limit point of $(s_n)$ then every subsequence of $(s_n)$ has a subsequence converging to $x$. By Problem 3.3.6 this would imply $s_n \to x$, which is a contradiction because we are assuming $(s_n)$ is non-convergent. We have shown that $(s_n)$ must have a limit point $x$ and that $x$ cannot be the only limit point, and therefore the only possibility is that $(s_n)$ has at least two limit points.
\end{proof}


\item (20 pts) \begin{itemize}
\item[(a)] Define $h:\R\to\R$ by $h(x)=x^3-2x^2+3x-6$. Carefully prove that the limit $\lim_{x\to 2}
h(x)$ exists.  (Your solution should also indicate the
limit!)

\begin{proof} Carefully, and cleverly, we will first prove results about the identity function and constant function.  Define the function $i:\R\to\R$ by $i(x)=x.$ First note that we have previously shown every $x\in\R$ is a limit point of $\R$, so we know $2\in\R$ is a limit point of the domain $\R$. To see that $\lim_{x\to 2} i(x) = 2,$ let $\epsilon>0.$ Then for $\delta=\epsilon$, if $0<|x-2|<\delta,$
$$|i(x)-2| = |x-2| <\delta = \epsilon.$$ Therefore $$\lim_{x\to 2} i(x) = 2.$$ Next fix $k\in\R$ and define $c_k:\R\to\R$ by $c_k(x)=k$. Let $\epsilon>0$ and pick $\delta=64.$ Then for $0<|x-2|<\delta,$ we have
$$|c_k(x)-k| = |k - k| = 0 < \epsilon.$$ Therefore $$\lim_{x\to 2} c_k(x) = k.$$ Next recall the function $h$ from above and observe that
\begin{align*} h(x) &= x^3-2x^2+3x-6\\
			        &= (i(x))^3 + c_{-2}(x)(i(x))^2 + c_3(x)i(x) + c_{-6}(x).
\end{align*} Apply Theorem 5.3.1 parts (1) and (3) a total of eight times (if you want me to be {\em really} careful, we apply part (1) three times and part (3) five times) to obtain
\begin{align*} \lim_{x\to 2} h(x) &=\lim_{x\to 2} \bigg((i(x))^3 + c_{-2}(x)(i(x))^2 + c_3(x)i(x) + c_{-6}(x)\bigg)\\
						&=(2)^3 + (-2)(2)^2 + (3)(2) + (-6)\\
						&=8-8 +6-6\\
						&=0.
\end{align*} Of course applying Theorem 5.3.1 guarantees that the limit does exist, and that it is equal to 0. \end{proof}

\item[(b)] Carefully prove that the limit $\lim_{x\to 0}\cos\frac{1}{x}$ fails to exist.

\begin{proof} We need to be quite careful indeed; the problem is actually stated quite ambiguously because it is not clear over which domain we are to consider this function. I think it is safe to assume that we are considering a function $f:\R^*\to \R$ defined by $f(x)=\cos\frac{1}{x}$, where $0\in\R$ is a limit point of the subset $\R^*$. To show $\lim_{x\to 0} f(x)$ does not exist, we will show that it is not equal to any candidate $L\in\R$. First consider $L=0$ and pick $\epsilon=\frac{1}{2}.$ Let $\delta>0$ and choose $n\in\N$ such that $n>\frac{1}{\delta}.$ Then $$0 < \Big|\frac{1}{2n\pi}-0\Big| <\frac{1}{n} < \delta,$$ yet
$$\left| f\left( \frac{1}{2n\pi}\right) - L \right| = \big| \cos(2n\pi) - 0 \big| = |1-0|
															= 1
															 > \epsilon.$$
This shows that $\lim_{x\to 0}f(x) \ne 0$. Now consider any $L\ne 0$, and pick $\epsilon=|L|.$ As before, for any $\delta>0$ choose $n\in\N$ such that $n>\frac{1}{\delta}.$ Then
$$0 < \Big|\frac{1}{2n\pi+\frac{\pi}{2}}-0\Big| <\frac{1}{n} < \delta,$$ yet
$$\left| f\left( \frac{1}{2n\pi+\frac{\pi}{2}}\right) - L \right| = \left| \cos\left(2n\pi+\frac{\pi}{2}\right) - L \right| = |0-L|=|L|=\epsilon.$$ Therefore $\lim_{x\to 0} f(x) \ne L$. So in considering the cases when $L=0$ and $L\ne0$, we have shown that the limit of $f$ as $x$ goes to 0 is not $L$ for any $L$ in the codomain $\R$. Therefore the limit does not exist.\end{proof}

\end{itemize}


\item  (15 pts) Let $(X,d)$ be a metric space.  Show that the
following statements about $X$ are equivalent.

\begin{description}
\item[i.]  $(X,d)$ is discrete. \item[ii.]  All functions on $X$
are continuous.  (The range can be {\em any} metric space!)
\end{description}

\begin{proof} \forward Suppose $(X,d)$ is discrete. Consider any $a\in X$. Since $\{a\}$ is open, there exists $r>0$ such that $B_r(a)\subseteq\{a\}$, from which it follows that $B_r(a)=\{a\}$. Then obviously $B_r(a)$ does not contain infinitely many points of $X$, so by the characterization in Theorem 3.5.1.3 we conclude that $a$ is not a limit point of $X$. Thus for any metric space $Y$ and any function $f:X\to Y$, $f$ is continuous at $a$ by definition. Since $a$ was arbitrary we conclude that all functions on X are continuous.

\back Conversely suppose that all functions on $X$ are continuous. Let $a\in X$ and define the function $f_a:X\to\R$ by
$$f_a(x) = \begin{cases} 64 & \text{ if } x=a\\
					0 & \text{ if }  x\ne a \end{cases} $$
Note that the open interval $(63,65) = B_1(64)$ is an open subset of $\R$. Since $f_a$ is continuous by hypothesis, Theorem 4.3.5 guarantees that the inverse image of $(63,65)$ is open in $X$. By the construction above we have $$f_a^{-1}\left((63,65)\right) = \{a\},$$ so the previous sentence implies that $\{a\}$ is open in $X$. Since $a$ was arbitrary, we conclude $X$ is a discrete metric space.
\end{proof}


\item (15 pts) Let $(a_n)$ and $(b_n)$ be Cauchy sequences in a metric space $(X,d)$.  Show that the sequence $(d(a_n, b_n))$ converges.  (Note that $X$ need not be complete, but $\R$ is complete!)

\begin{proof} As hinted above, we will show that $(d(a_n, b_n))$ is a Cauchy sequence and invoke the completeness of $\R$ to complete the proof. Let $\epsilon>0$. Since $(a_n)$ and $(b_n)$ are Cauchy sequences, we can choose $N_1\in\N$ and $N_2\in\N$ so that 
$$d(a_n,a_m)<\frac{\epsilon}{2}\quad\text{ for all}\; n,m>N_1$$
$$d(b_n,b_m)<\frac{\epsilon}{2}\quad\text{ for all}\; n,m>N_2.$$ Let $N=\max\{N_1,N_2\}$ and $n,m>N$. Supposing without loss of generality that $d(a_n,b_n)\ge d(a_m,b_m)$,
\begin{align*} \left| d(a_n,b_n) - d(a_m,b_m)\right| &= d(a_n,b_n) - d(a_m,b_m)\\
										&\le  d(a_n,a_m) + d(a_m,b_n) - d(a_m,b_m)\\
							&\le d(a_n,a_m) + d(b_n,b_m) +d(b_m,a_m) - d(a_m,b_m)\\
									&=d(a_n,a_m) + d(b_n,b_m) \\
									&<\frac{\epsilon}{2}+\frac{\epsilon}{2} = \epsilon.\end{align*} Therefore $(d(a_n,b_n))$ is a Cauchy sequence in $\R$ and since $\R$ is complete, $(d(a_n,b_n))$ converges.\end{proof}

\item (20 pts)
\begin{itemize}
\item[(a)] Let $(f_n)$ be a sequence of real-valued continuous
functions on a metric space $X$.  Suppose that $(f_n)$ converges
uniformly to a function $f$.  Prove that if all terms (i.e. all
functions) of the sequence $(f_n)$ are bounded, then $f$ is
bounded.

\begin{proof} Let $X$ be a metric space and $f_n: X\to\R$ be a sequence of bounded functions that converges uniformly to $f$. To show that $f$ is bounded we need to show that the range $f(X)=\{f(x): x\in X\}$ is bounded in $\R$, and to do so we will use the characterization given in Corollary 3.1.14. Explicitly, we will show that there exists $M\in\R$ such that $|f(x)|\le M$ for all elements $f(x)\in f(X).$ We will need to use the uniform convergence assumption, so let $\epsilon=64$. Since $(f_n)$ converges uniformly to $f$, we can choose $N\in\N$ so that for all $n>N$ and all $x\in X$, $ |f_n(x)-f(x)| < 64$. Now fix an $n>N$. Since $f_n$ is bounded, by Corollary 3.1.14 there exists $K\in\R$ such that $|f_n(x)|\le K$ for all $x\in X$. Then for any $x\in X$,
\begin{align*} |f(x)| = |f(x)-0| &\le |f(x)-f_n(x)| + |f_n(x) - 0|\\
					&<64+|f_n(x)|\\
					&\le 64+K.
\end{align*} Therefore $|f(x)| \le 64+K$ for all $x\in X$, so $f(X)$ is bounded in $\R$. By definition, $f$ is bounded.
\end{proof}

\item[(b)] Recall that a function $f:X\to Y$ is said to be
\textbf{bounded} if its range $f(X)$ is a bounded subset of $Y$.
Give an example to show that the pointwise limit of real-valued
bounded functions on a metric space $X$ need not be a bounded
function.  Hint: consider piecewise defined functions on
$\mathbb{R}$.\\

\begin{example} Consider $\N$ under the Euclidean metric and the sequence of functions $f_n:\N\to\R$ defined by
$$f_n(m) = \begin{cases} m & \text{ if } m\le n\\
					0 & \text{ if } m>n. \end{cases}$$
Now each $f_n$ is bounded by $n$:
\begin{align*} f_n(\N) &= \{1, 2, 3, \ldots, n, 0 , 0,0,...\}\\ &= \{0,1,2,\ldots,n\}.\end{align*}
However, it is clear that $(f_n)$ converges pointwise to the function $f(m)=m.$ To show this rigorously, let $m\in\N$ (as a point in the domain!) and $\epsilon>0$. Then for all $n>m$ (as a sequence index!),
$$|f_n(m)- f(m)| = |m-m| = 0 < \epsilon.$$
Therefore $f_n(m) \to f(m)$ at the point $m$, and since $m$ was arbitrary we have shown that $(f_n)$ converges pointwise to $f$. But obviously $f(\N)=\N$ is unbounded! Therefore uniform convergence is in fact a necessary condition for part (a).\\

Of course, using the same idea I could have constructed $f_n:\R\to\R$ as
$$f_n(x) = \begin{cases} |x| & \text{ if } x\le n\\
					0 & \text{ if } x>n \end{cases}$$
and gotten the same result. This would have even avoided confusion with regards to choosing $m\in\N$ in the domain and shortly thereafter using $N=m$ for the sequence index that we need in showing a sequence converges. However, I like the way I did it because $\N$ is 1-separated under the Euclidean metric and thus has no limit points. So the sequence functions $f_n$, while piecewise, are actually continuous.\end{example}

\end{itemize}

\end{enumerate}




\end{document}








\noindent \textbf{Directions:}\\

\noindent This test is due at \textbf{5:00pm on Wednesday December
19,
2012.}\\

\noindent On this exam, you are not to use any resources besides
your textbook and your class notes.  This means, among other
things, you may not consult any other books, any online sources,
or any people besides your instructor.  You may ask me clarifying
questions at any point.\\

\noindent You are also responsible for taking reasonable
precautions to make sure your notes, drafts, and scratchwork are
not accessible to anyone besides you \textbf{until 5 pm Friday,
December 21}.  This means, among other things, that you should not
work on any whiteboards in public areas, that you should not leave
paper with your work in cubbies or on tables in public areas, that
you should not leave drafts on unattended computer screens or
printouts unattended on public printers, and that you should not
discard paper with your notes or draft work in public
trash/recycle bins until after the
exam has been collected.\\

\noindent There is no time limit for this exam.\\

\noindent You are required to typeset your solutions in some
format that appropriately displays the mathematics at hand. For
your convenience I will place a file with the LaTeX code for this
exam on the P drive.\\

\noindent Note: these problems are not necessarily ordered by
level of difficulty, and you should work on them ``in
parallel.''\\

\noindent Good luck, and have fun!

\newpage
